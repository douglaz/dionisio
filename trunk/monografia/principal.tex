\documentclass[a4paper,12pt]{report}

\usepackage[brazilian,english]{babel}
\usepackage{indentfirst}
\usepackage{a4wide}
\usepackage[section]{placeins}
\usepackage[utf8]{inputenc}
% \usepackage[Conny]{fncychap}
\usepackage[T1]{fontenc}
\usepackage{txfonts}
\usepackage{type1cm}
\usepackage{courier}
\usepackage[scaled]{helvet}
\usepackage{epigraph}
\renewcommand*\familydefault{\sfdefault}

\hyphenation{
  es-ta-be-le-ci-do
  a-de-qua-da-men-te
  pro-ble-mas
  di-men-sio-na-men-to
  mo-de-lo
}

% Controlar linhas orfas e viuvas
\clubpenalty=10000
\widowpenalty=10000
\displaywidowpenalty=10000

% Formatar notas de rodape
\usepackage[hang]{footmisc}
\setlength{\footnotemargin}{1em}

\usepackage{url}
%% Define a new 'leo' style for the package that will use a smaller font.
\makeatletter
\def\url@leostyle{%
  \@ifundefined{selectfont}{\def\UrlFont{\sf}}{\def\UrlFont{\small\ttfamily}}}
\makeatother
%% Now actually use the newly defined style.
\urlstyle{leo}

\usepackage[left=3cm,top=3cm,right=2cm,bottom=2cm,includehead,ignoremp]{geometry}

\usepackage[small]{titlesec}
% \titlespacing{\chapter}{0pt}{-50pt}{*20}[0pt]
\titlespacing{\chapter}{0pt}{*-10}{*5}

% Running Headers and footers
\usepackage{fancyhdr}
% \pagestyle{fancy}
\pagestyle{headings}
% Redefine plain page style
\fancypagestyle{plain}{
  \fancyhf{}
  \renewcommand{\headrulewidth}{0pt}
  \fancyhead[LE,RO]{\thepage}
}

\linespread{1.3}
%\pdfpagewidth=\paperwidth
%\pdfpageheight=\paperheight

% Para habilitar cálculos em dimensões
\usepackage{calc}

% Multipart figures
%\usepackage{subfigure}

% More symbols
%\usepackage{amsmath}
%\usepackage{amssymb}
%\usepackage{latexsym}

% Surround parts of graphics with box
\usepackage{boxedminipage}

% Package for including code in the document
\usepackage{listings}

% If you want to generate a toc for each chapter (use with book)
%\usepackage{minitoc}

% This is now the recommended way for checking for PDFLaTeX:
\usepackage{ifpdf}

\ifpdf
\usepackage[pdftex]{graphicx}
\else
\usepackage{graphicx}
\fi

\title{Dionisio}
\author{Allan Douglas R. de Oliveira \\ allandouglas@gmail.com \and
Leonardo Nicacio Bessa \\ leobessa@gmail.com \and
Thiago Rodrigues Andrade \\ suffragium@gmail.com}
\date{2009-04-22}

\usepackage[absolute]{textpos}
\begin{document}

\ifpdf
\DeclareGraphicsExtensions{.png, .pdf, .jpg, .tif}
\else
\DeclareGraphicsExtensions{.eps, .jpg}
\fi

%\maketitle
\pagestyle{empty}
\begin{center}
  {\large ALLAN DOUGLAS R. DE OLIVEIRA \\
    LEONARDO NICACIO BESSA \\
    THIAGO RODRIGUES ANDRADE}
\end{center}

\begin{textblock*}{21cm}[0,0](0cm,12cm)
  \begin{center}
    {\LARGE Dionisio:\\ Um sistema de recomendação baseado em confiança }
  \end{center}
\end{textblock*}


\begin{textblock*}{21cm}[0,0](0cm,29.7cm-4cm)
  \begin{center}
    {\large São Paulo \\ 2009 }
  \end{center}
\end{textblock*}

\newpage
\begin{center}
  {\large ALLAN DOUGLAS R. DE OLIVEIRA \\
    LEONARDO NICACIO BESSA \\
    THIAGO RODRIGUES ANDRADE}
\end{center}

\begin{textblock*}{21cm}[0,0](0cm,12cm)
  \begin{center}
    {\LARGE Dionisio:\\ Um sistema de recomendação baseado em confiança }
  \end{center}
\end{textblock*}


\begin{textblock*}{21cm}[0,0](0cm,29.7cm-4cm)
  \begin{center}
    {\large São Paulo \\ 2009 }
  \end{center}
\end{textblock*}


\begin{textblock*}{21cm/2-2cm}[0,0](21cm/2,15cm)
  \begin{flushleft}
    {\large
  Monografia apresentada à Escola Politécnica da Universidade de São Paulo para obtenção do título de Bacharel em Engenharia.\newline
  \newline
  Área de Concentração:\newline
  Engenharia de Computação\newline
}
  \end{flushleft}
\end{textblock*}



\begin{center}
  {\large ALLAN DOUGLAS R. DE OLIVEIRA \\
    LEONARDO NICACIO BESSA \\
    THIAGO RODRIGUES ANDRADE}
\end{center}

\begin{textblock*}{21cm}[0,0](0cm,12cm)
  \begin{center}
    {\LARGE Dionisio:\\ Um sistema de recomendação baseado em confiança }
  \end{center}
\end{textblock*}


\begin{textblock*}{21cm}[0,0](0cm,29.7cm-4cm)
  \begin{center}
    {\large São Paulo \\ 2009 }
  \end{center}
\end{textblock*}


\begin{textblock*}{21cm/2-2cm}[0,0](21cm/2,15cm)
  \begin{flushleft}
    {\large
  Monografia apresentada à Escola Politécnica da Universidade de São Paulo para obtenção do título de Bacharel em Engenharia.\newline
  \newline
  Área de Concentração:\newline
  Engenharia de Computação\newline
}

    {\large
      Orientador: Prof. Dr. Jaime Simão Sichman \\
        Co-orientadora: Prof. Dra. Lucia Filgueiras
    }
  \end{flushleft}
\end{textblock*}


%\begin{textblock*}{21cm/3}[0,0](21cm/3*2-2cm,29.7cm-6cm)
  \begin{flushright}
    {\emph{
           Aos nossos pais e companheiras, pelo apoio e incentivo em nossa formação pessoal. Aos mestres, pelo conhecimento transmitido e pela paciência que tiveram conosco.
          }
    }
  \end{flushright}
\end{textblock*}

\null\newpage


% agradecimentos (opcional)
%\begin{textblock*}{21cm/3}[0,0](21cm/3*2-2cm,29.7cm-6cm)
  \begin{flushright}
\epigraph{\emph{``Innovation distinguishes between a leader and a follower.''}}{Steven Paul Jobs\\Co-fundador da Apple Inc.}
\end{flushright}
\end{textblock*}

\null\newpage


\pagenumbering{roman}

% Resumo e Abstract
\selectlanguage{brazilian}
%  \begin{abstract}
%    O comércio eletrônico brasileiro tem evoluído em termos de serviços prestados ao consumidor. Este trabalho apresenta a especificação e implementação de um sistema online de recomendação para facilitar o encontro do usuário com os produtos e serviços mais relevantes a ele. Para isso, são descritos em detalhes os requisitos funcionais e não funcionais do sistema, os algoritmos de recomendação utilizados e as escolhas na implementação do projeto. 
O sistema projetado é composto por três algoritmos de recomendação, sendo dois deles já difundidos, enquanto o terceiro é um novo, e que foi proposto para levar em consideração as relações de confiança do usuário com seus pares. 
A execução deste trabalho incluiu a realização de um experimento com pessoas interagindo com o sistema de recomendação através de uma rede social. A análise do experimento é apresentada através de gráficos e dados estatísticos que comparam os algoritmos utilizados. 
Os resultados do projeto podem ser aplicados no futuro no desenvolvimento de lojas virtuais online e/ou aplicativos de redes sociais.
%  \end{abstract}
%\begin{otherlanguage}{english}
%  \begin{abstract}
%    The Brazilian e-commerce has evolved in terms of services provided to the consumer. This paper presents the specification and implementation of an online recommendation system that eases the matching of a user with the most relevant products and services. Furthermore the functional and non-functional system requirements, the selected recommendation algorithms and the choices made in the implementation are described in detail.
The designed system consists of three recommendation algorithms, two of them are already widely spreaded, while the third one is new and was proposed to consider the trust among users and their peers.
This work also includes an experiment with people interacting with the recommendation system through a social network. The analysis of the experiment is presented through graphs and statistics that compare the algorithms.
The results of the project can be applied in future development of virtual online stores and/or social networking applications.
%  \end{abstract}
%\end{otherlanguage}

%\listoffigures \thispagestyle{headings} % Lista de Figuras

% \listoftables \thispagestyle{empty} % Lista de Tabelas

\tableofcontents \thispagestyle{headings} % Sumario

% lista de abreviaturas e siglas (opcional)
% lista de simbolos (opcional)

\pagestyle{headings}

% Elementos de Texto
\chapter{INTRODUÇÃO} \pagenumbering{arabic} % (fold)

\section{Objetivo} % (fold)
\label{sec:objetivo}
% Apresentar de forma precisa e concisa o objetivo do projeto.

 O principal objetivo deste projeto é criar um sistema de recomendação baseado em recursos disponíveis na Web, que possibilite a sugestão de itens confiáveis e relevantes ao usuário.

% section objetivo (end)

\section{Motivação} % (fold)
\label{sec:motivação}

% O que é importante para: nós, comunidade científica, usuários, mercado.

% Apresentar a motivação e justificativa para a realização do trabalho (por exemplo, sua aplicabilidade prática, comparação com alternativas já existentes, potencial de  aprendizado e evolução, etc).

% http://www.uxmatters.com/mt/archives/2009/03/including-recommendations-in-user-interfaces-to-enhance-motivation.phps

 Sistemas de Recomendação sugerem aos usuários itens que eles possam gostar, baseado no comportamento prévio do usuário. Fazendo suposições pertinentes sobre o tipo de objetos em que os usuários estão interessados, é possível conquistar a sua confiança. A vantagem para os usuários é a facilidade de encontrar a informação, sem ter a árdua tarefa de procurá-la.

 As redes sociais online têm modificado a forma com que as empresas utilizam a comunicação para o comércio. Pessoas estão utilizando a Web para encontrar outras pessoas com interesses similares, fazer compras de forma mais eficiente, aprender sobre produtos e serviços e reclamar sobre produtos malfeitos\cite{marketing_social_web}.

 A Web está rapidamente se tornando a mídia mais importante para o marketing. A tendência é que as pessoas cada vez mais bloqueiem os anúncios indesejados e queiram ter a capacidade de encontrar os produtos relevantes no momento adequado. É nesse contexto que surge a necessidade de uma plataforma que facilite a colaboração e que permita a criação e classificação de conteúdo pelos consumidores, de forma a permitir uma escolha mais inteligente dos melhores produtos e serviços, e ao mesmo tempo criando uma mecanismo de feedback para as empresas interessadas.

 Devido à grande variedade atual de produtos e serviços, as pessoas têm cada vez mais dificuldade nas suas escolhas e na argumentação sobre a possível decisão. Quanto maior for a quantidade de produtos similares de fabricantes diferentes, mais as pessoas se vêem desnorteadas e sem saber se a decisão realizada foi a mais correta.
 
 % TODO: Escrever sobre Web Semântica

% paragraph paragraph_name (end)
% section motivação (end)

%\section{Organização} % (fold)
%\label{sec:organização}

% Não fazer ainda!!!
% Apresentar a organização do documento: o que cada capítulo, anexo e apêndice aborda.


% section organização (end)

% chapter web_social (end)
\chapter{SISTEMAS DE RECOMENDAÇÃO} \pagenumbering{arabic}% (fold)
\label{cha:sistemas_de_recomendação}

\section{Introdução}
Sistemas de recomendação~\footnote{Em inglês, \textit{Recommender systems}.} são aqueles que sugerem itens ao seus usuários de forma a ajudá-los a encontrar mais efetivamente os itens de maior interesse dentre uma variedade imensa de opções.

A idéia de recomendar itens é algo que pode ser observado no dia-a-dia das pessoas. Como muitas escolhas precisam ser feitas sem que se tenha uma experiência pessoal das alternativas, as pessoas se baseiam no que as outras dizem sobre um determinado produto antes de comprá-lo ou experimentá-lo. Estas informações são transmitidas boca-a-boca entre amigos e colegas ou através de resenhas especializadas que podem encontradas em revistas e jornais.

Os sistemas de recomendação auxiliam este processo social, agregando opiniões e avaliações de uma comunidade de usuários sobre os produtos e recomendando itens de acordo com o perfil do usuário desta comunidade.

\section{Contexto histórico}
O primeiro sistema de recomendação, Tapestry~\footnote{Ver \cite{Goldberg92}.}, foi desenvolvido no início da década de 90.~\cite{Resnick97} Na última década o sistemas, principalmente os baseado em filtragem colaborativa, foram um grande foco de estudo~\cite{Herlocker04}.

Graças ao acumulo de informações provenientes da web social, observa-se hoje que vários sistemas de recomendação podem ser experimentados pelo usuários da Internet, como exemplo:

\begin{itemize}
\item 
A Amazon~\footnote{http://www.amazon.com} e o Submarino~\footnote{http://www.submarino.com.br.}, lojas virtuais de artigos diversos, recomendam itens semelhantes para aqueles usuários que compram um produto ou manifestam interesse em comprá-lo.

\item O Last.fm~\footnote{http://last.fm.}, uma rede social focada em música, recomenda artistas e canções semelhantes àquelas que os usuários mais gostam de ouvir.

\item O Digg~\footnote{http://digg.com.} e o Delicious~\footnote{http://del.icio.us.}, sistemas de compartilhamento de links~\footnote{Também conhecidos como bookmarks.}, geram uma lista geral de links recomendados baseado nas opiniões dos usuários do sistema.

\item O StumbleUpon~\footnote{http://stumbleupon.com.}, também um sistema online de compartilhamento de links, permite que os usuários recebam recomendações de links e avaliem se eles gostaram ou não daquele link, gerando recomendações personalizadas baseadas nessa avaliação.
\end{itemize}


\section{Classificações dos sistemas de recomendação}

Nas próximas seções serão apresentadas as diferentes abordagens para a implementação de sistemas de recomendação. As primeiras abordagens foram a filtragem colaborativa e a filtragem baseada em conteúdo. Vários sistemas são híbridos, isto é, utilizam mais de uma abordagem.

%Na seção (seção XXX) serão discutidos as diferentes implementações encontradas na literatura.

\section{Filtragem baseada em conteúdo} % (fold)
A filtragem baseada em conteúdo consiste na extração de \textit{features} dos itens a serem recomendados e da comparação desses \textit{features} com aqueles que formam o  perfil histórico do usuário. Esse é um dos primeiros métodos que surgiram e sua origem está na comunidade de \textit{information retrieval}.~\cite{Balabanovi97}

Como exemplo, suponha-se que se queira recomendar documentos em formato texto. Os \textit{features} nesse caso poderiam ser as palavras do texto. O perfil histórico do usuário seria formado pela frequência acumulada das palavras presentes em cada texto avaliado pelo usuário. Um documento neste caso é recomendado se os \textit{features} (palavras) presentes podem ser encontrados em grande frequência nos documentos avaliados positivamente pelo usuário no passado.

Outros exemplos de \textit{features} que podem ser usados em um documento são meta-informações como autor, categoria do documento (artigo, jornal, revista, por exemplo), assunto (computação, matemática, artes, esportes), entre outras palavras-chave.

% (falar um pouquinho porque é bom/ruim)

\section{Filtragem social} % (fold)

% Talvez usar o Pazzani96syskill para falar de content-based

A filtragem social\footnote{Termo originado em \cite{Malone87} segundo \cite{Hill95}.} consiste em um conjunto de técnicas que utilizam o contexto e as relações sociais de uma comunidade de usuários para fazer recomendações. Ao contrário da filtragem colaborativa, o conteúdo de cada item não é analisado, possibilitando-se recomendar qualquer tipo de item.

\subsection{Filtragem colaborativa}

O termo \textit{collaborative filtering} foi cunhado por \cite{Goldberg92}~\footnote{Conforme \cite{Resnick97}.}. A abordagem básica consiste em montar um sistema que permite os usuários fazerem avaliações\footnote{Em inglês, \textit{ratings} ou \textit{votes}.} dos itens que podem ser recomendados. Isso resulta em triplas (usuário, item, avaliação). A partir dessas avaliações, pode-se criar uma lista de recomendação para um determinado usuário. Na descrição a seguir, este usuário a quem se deseja recomendar algo será chamado usuário ativo. A avaliação de um item feita por um usuário será chamada de voto. Os passos são os seguintes:

\begin{enumerate}
\item 
Para cada usuário do sistema, determina-se a similaridade entre este usuário e o usuário ativo. Este valor será designado por $s$.

\item Para cada item do sistema não avaliado pelo usuário ativo, calcula-se o voto previsto $p$ para aquele item utilizando-se o voto de cada usuário que avaliou o item e a similaridade $s$ entre o usuário e o usuário ativo. O voto previsto é uma estimativa da avaliação que aquele usuário faria do item caso já o conhecesse.
\end{enumerate}

\paragraph{Geração de uma lista de recomendação de tamanho $n$.}

A lista de itens recomendados será formada pelos $n$ itens com os maiores votos previstos ordenados decrescentemente.

\subsubsection{Formulação matemática}
Mais detalhadamente, o algoritmo básico é o seguinte:

Sendo $v_{i,j}$ o voto do usuário $i$ para o item $j$, $I_{i}$ o conjunto de itens que foram avaliados pelo usuário $i$, define-se voto médio de um usuário $i$ por:

\begin{equation}
 \bar{v_{i}} = \frac{1}{|I_{i}|} \sum_{j \in I_{i}} v_{i,j}
\end{equation}

O valor previsto do voto do usuário ativo $a$ para o item $j$, será dado por:

\begin{equation}
 p_{a,j} = \bar{v_{a}} + k\sum_{i=1}^n{s(a,i) (v_{i,j} - \bar{v_{i})}}
 \label{eq:filtragem_colaborativa_similaridade} 
\end{equation}
onde $n$ é o número de usuários do sistema, $s(a,i)$ é a similaridade entre o usuário $a$ e o usuário $i$ e $k$ é um fator de normalização, dado neste caso por:

\begin{equation}
 k = \sum_{i=1}^n{\frac{1}{s(a,i)}}
\end{equation}


\paragraph{Cálculo de $s$.}

O cálculo de $s$ geralmente é realizado utilizando-se o coeficiente de correlação de Pearson\cite{Breese98}, definido por:

\begin{equation}
 s(a,i) = \frac{\sum_{j}{(v_{a,j} - \bar{v_{a}}) (v_{i,j} - \bar{v_{i}})}}{\sqrt{\sum_{j}{(v_{a,j} - \bar{v_{a}})}^2\sum_{j}{(v_{i,j} - \bar{v_{i}})}^2}}
\end{equation}

\paragraph{Forma geral da Filtragem Colaborativa.}

Apesar dos primeiras abordagens terem utilizado a similaridade $s$, qualquer peso $w$\footnote{Do inglês \textit{weigth}} pode ser usado na equação \ref{eq:filtragem_colaborativa_similaridade}. Pode-se então reescreve-la da seguinte maneira geral:

\begin{equation}
 p_{a,j} = \bar{v_{a}} + k\sum_{i=1}^n{w(a,i) (v_{i,j} - \bar{v_{i})}}
 \label{eq:filtragem_colaborativa_geral} 
\end{equation}

Para mais detalhes, ver \cite{Breese98}.
% Explicar com mais detalhes o método e citar outros tipos de w


\subsection{Filtragem baseada em confiança} % (fold)

A filtragem baseada em confiança\footnote{Entendida aqui no contexto de \textit{Trust-aware recommender systems}~\cite{Massa07}} é muito similar à filtragem colaborativa, mas utiliza como peso $w(a,i)$ a confiança do usuário $a$ no usuário $i$.

Esta confiança é fornecida de forma explícita pelo usuário. Este método se mostrou muito efetivo no casos em que a maioria dos usuários do sistema são \textit{cold-users}, isto é, quando a maioria dos usuários avaliou poucos itens e portanto possuem poucos itens em comum. Em um cenário como esse a filtragem colaborativa tradicional falha em encontrar usuários semelhantes pois há pouca intersecção de avaliações de itens~\footnote{Em inglês, \textit{rating-overlap}.}.

%\section{

%\section{Discussão sobre as diferentes abordagens}



\chapter{WEB SOCIAL} % (fold)
\label{cha:web_social}

% Qual é a melhor solução neste contexto?

% Por que ela depende de recomendação?

% chapter web_social (end)
\chapter{ESPECIFICAÇÃO DO PROJETO} % (fold)
\label{cha:especificacao_do_projeto}

\section{Funcionalidades principais} % (fold)
\label{sec:funcionalidades_principais}

% section funcionalidades_principais (end)

\section{Tecnologia} % (fold)
\label{sec:tecnologia}

% section tecnologia (end)

\section{Planejamento e Métodos} % (fold)
\label{sec:planejamento_e_métodos}


% section planejamento_e_métodos (end)

% chapter especificacao_do_projeto (end)
%\include{capitulos/5-metodologia}
%\include{capitulos/6-implementacao}
%\include{capitulos/7-testes}
%\include{capitulos/8-conclusoes}

\addcontentsline{toc}{chapter}{Referências Bibliográficas}
\bibliographystyle{abnt-num}   %\bibliographystyle{plain}
% \bibliography{bibname.bib} % bibname=nome do seu arquivo BibTeX
\begin{thebibliography}{99}

 

%  \bibitem{marketing_social_web}
%    WEBER, LARRY.
%    \textbf{Marketing to the social web}: how digital customer communities build your business.
%    Wiley, 2007, 5 p.

\end{thebibliography}


%\addcontentsline{toc}{chapter}{Glossário}
%\chapter*{Glossário} % (fold)
\label{cha:glossario}

\begin{itemize}

  \item \emph{ADSL}: acrônimo de \emph{Asymmetric Digital Subscriber Line}, é uma forma de transmissão de dados de alta velocidade utilizando linhas telefônicas comuns, em freqüências maiores que os seres humanos conseguem escutar.

  \item \emph{BNC}: é um tipo de conector cujo nome vem de seus criadores: ``bayonet Neil-Concelman''. Sua grande característica é o sistema de trava, tipo twist-lock (gira e trava), que possibilita grande segurança na conexão. É bastante utilizado nos equipamentos profissionais de vídeo.

  \item \emph{Broadcast}: é um modo de difusão de sinais em que é transmitido o mesmo conteúdo para todos os receptores. Numa transmissão de TV, por exemplo, todas as pessoas sintonizadas no mesmo canal assistem ao mesmo programa. Em Internet, o termo é usado muitas vezes para designar o envio de uma mensagem par todos os membros de um grupo, em vez da remessa para membros específicos.

  \item \emph{Buffer}: é uma área de armazenamento que compensa diferentes velocidades de fluxos de dados ou temporizações de eventos, ao transferir dados de um dispositivo para outro.

  \item \emph{CD}: acrônimo de Compact Disc, é um padrão de armazenamento óptico para dados digitais.
  Conversor A/D: um Conversor Analógico-Digital é componente de um sistema responsável por converter dados analógicos para digitais através da amostragem de um sinal contínuo e sua posterior discretização gerando valores numéricos digitais.

  \item \emph{Classpath}: é um argumento passado para a Java Virtual Machine indicando onde procurar classes e pacotes para carregamento dinâmico.

  \item \emph{CPqD}: {C}entro de {P}es{q}uisa e {D}esenvolvimento em Telecomunicações.

  \item \emph{DivX}: é um formato de compactação de vídeo criado pela DivxNetworks Inc.

  \item \emph{Driver}: um driver é um componente de software responsável por estabelecer a comunicação entre hardware e software, provendo comandos para enviar e receber dados de um dispositivo instalado.

  \item \emph{DVD}: acrônimo de Digital Versatile Disc, é a geração seguinte ao CD, possibilitando um armazenamento maior de dados.

  \item \emph{ECMAScript}: é uma linguagem de programação baseada em scripts, padronizada pela Ecma International na especificação ECMA-262. A linguagem é bastante usada em tecnologias para Internet, sendo esta base para a criação do JavaScript/JScript e também do ActionScript.

  \item \emph{FIFO}: acrônimo para {F}irst {I}n, {F}irst {O}ut (que em português significa primeiro a entrar, primeiro a sair) refere-se a estruturas de dados do tipo fila onde os elementos vão sendo colocados e retirados (ou processados) por ordem de chegada.

  \item \emph{GEM}: acrônimo de \emph{Globally Executable MHP} \cite{gem}, é uma parte do MHP independente dos padrões de transmissão europeus, criado com a finalidade de padronizar partes de todos os padrões de \emph{middleware} mundiais e possibilitar a criação de aplicações que funcionem em qualquer \emph{middleware} que seja compatível com o GEM.

  \item \emph{GSM}: acrônimo de Global System for Mobile Communications é um dos principais padrões para telefonia móvel existente.

  \item \emph{Heap}: Heap de objetos é o nome dado a parte da memória do computador que contém a estrutura de dados responsável por armazenar todos os objetos durante a execução da máquina virtual Java.

  \item \emph{HTTP}: acrônimo para {H}yperText {T}ransfer {P}rotocol (Protocolo de Transferência de Hipertexto), utilizado para transferência de dados na rede mundial de computadores, a World Wide Web.

  \item \emph{IPC (Inter-Process Communication)}: é o grupo de mecanismos que permite aos processos transferirem informação entre si. Entre estes mecanismos podem ser citados pipes, filas de mensagens e memória compartilhada.

  \item \emph{Java}: é uma linguagem de programação orientada a objeto desenvolvida na década de 90 pelo programador James Gosling, na empresa Sun Microsystems.

  \item \emph{JavaBeans}: são componentes reutilizáveis de software escritos em linguagem Java, e que seguem algumas convenções de modo a permitir que ferramentas possam utilizá-los e manipulá-los.

  \item \emph{JavaTV}: é uma biblioteca Java que contempla a maior parte dos recursos necessários para a operação de sistemas receptores de TV digital, simplificando assim o desenvolvimento de softwares, uma vez que os programadores de aplicativos podem se voltar ao tema principal da aplicação em desenvolvimento.

  \item \emph{JNI (Java Native Interface)}: é um padrão de programação que permite que a máquina virtual da linguagem Java acesse bibliotecas construídas com o código nativo de um sistema.

  \item \emph{Lista ligada}: é uma estrutura de dados linear e dinâmica composta por células que apontam para o próximo elemento da lista.

  \item \emph{Metodologia (Processo) de Desenvolvimento Ágil}: metodologias de desenvolvimento ágil foram pensadas de forma a minimizar riscos no desenvolvimento de software através períodos mais curtos de lançamento, chamados iterações, que tipicamente levam de uma quatro semanas. Métodos ágeis prezam mais a comunicação face-a-face que a documentação, como forma de acelerar o processo de desenvolvimento.

  \item \emph{MHP}: acrônimo de \emph{Multimedia Home Platform} \cite{mhp}, é um padrão aberto de \emph{middleware} para TV Digital, adotado principalmente pelos países europeus. Está incluso em outro padrão maior, o \emph{Digital Video Broadcasting} (DVB) \cite{dvb} que agrupa todas as características da tecnologia de TV Digital européia.

  \item \emph{Middleware} (conforme \cite{ginga}): é a camada de software intermediário que permite o desenvolvimento de aplicações interativas para a TV Digital de forma independente da plataforma de hardware dos fabricantes de terminais de acesso (\emph{set-top boxes}).

  \item \emph{MPEG}: acrônimo para {M}oving {P}icture {E}xperts {G}roup, é o grupo de trabalho da Organização Internacional para Padronização (ISO) para o desenvolvimento de padrões para vídeo e áudio digitais.

  \item \emph{MPEG2}: é um padrão de compressão e codificação de vídeo para difusão e comunicações, bem como para armazenamento em meios diversos, tais quais os ópticos.

  \item \emph{MPEG-TS}: MPEG {T}ransport {S}tream (TS, TP, ou MPEG-TS) é um protocolo de  comunicação para transmissão de áudio, vídeo e dados, especificado pelo padrão  ISO/IEC 13818-1. Permite multiplexar vídeo e áudio digital sincronizando a saída. Possui correção de erro e transporte, e é usado para difusão de aplicações como DVB e ATSC.

  \item \emph{MOV}: formato de vídeo criado pela Apple, é um container para imagem, diversas trilhas, efeitos e textos. Sua base foi aprovada pela ISO como padrão para MPEG4 Part. 14.

  \item \emph{Multiplexação no tempo}: a multiplexação no tempo consiste na transmissão de dois ou mais sinais ou fontes de bits simultaneamente através de um único canal pela divisão do tempo em pequenos compartimentos de tamanhos fixos, onde são transmitidos alternadamente um pouco de cada sinal.

  \item \emph{Overhead}: Em computação overhead é geralmente considerado qualquer processamento ou armazenamento em excesso, seja de tempo de computação, de memória, de largura de banda ou qualquer outro recurso que seja requerido para ser utilizado ou gasto para executar uma determinada tarefa.

  \item \emph{Pipe (UNIX)}: é o redirecionamento da saída padrão de um programa para a entrada padrão de outro.

  \item \emph{{S}et-{t}op {B}ox ({STB})}: é o termo que descreve um equipamento que se conecta a um televisor e a uma fonte externa de sinal, transformando este sinal em conteúdo no formato que possa ser apresentado em uma tela.

  \item \emph{SBTVD} \cite{sbtvd}: acrônimo para {S}istema {B}rasileiro de {TV} {D}igital.

  \item \emph{SMS}: acrônimo para {S}hort {M}essage {S}ervice. Tecnologia amplamente utilizada em telefonia celular para a transmissão de mensagens de texto curtas.

  \item \emph{SOAP}: acrônimo de \emph{Service Oriented Architecture Protocol}, é um protocolo de troca de mensagens em formato XML.

  \item \emph{Thread}: é uma forma de um processo dividir a si mesmo em duas ou mais tarefas que podem ser executadas simultaneamente.

  \item \emph{UML}: acrônimo de Unified Modelling Language, é um padrão gráfico de especificação para modelagem de objetos, e uma ferramenta importante no processo de desenvolvimento de software.

  \item \emph{{USB}}: acrônimo de {U}niversal {S}erial {B}us, é uma especificação para interfaces de comunicação serial de dados. É padronizado pelo {USB} {I}mplementers {F}orum ({USB-IF}), que possui membros como Apple Inc., Hewlett-Packard, NEC, Microsoft e Intel.

  \item \emph{W3C}: acrônimo de World Wide Web Consortium, é o principal órgão de padronização para a Web (Internet).

  \item \emph{Web Services}: é definido pelo W3C como um sistema de software projetado para dar suporte à comunicação interoperável entre duas máquinas utilizando uma rede. Essa comunicação é feita através de mensagens XML utilizando-se servidores web, em um padrão denominado SOAP.

  \item \emph{Wi-Fi}: é uma marca registrada pertencente à Wireless Ethernet Compatibility Alliance (WECA), e é a abreviatura de \emph{wireless fidelity}, sendo uma tecnologia de interconexão entre dispositivos sem fio, usando o protocolo IEEE 802.11.

  \item \emph{WiMax}: acrônimo de \emph{Worldwide Interoperability for Microwave Access}, é o nome comercial para o padrão IEEE 802.16, que especifica uma interface sem fio para redes metropolitanas, agregando conhecimentos e recursos mais recentes ao padrão Wi-Fi visando melhor performance de comunicação.

  \item \emph{WMV}: \emph{Windows Media Video} é o nome para uma série de formatos de vídeo compactados criados pela Microsoft.

  \item \emph{XHTML}: acrônimo para \emph{eXtensible HyperText Markup Language}, é uma linguagem de marcação com as mesmas marcas do HTML, porém com uma sintaxe mais rigorosa pois é baseada no XML, e dessa podem ser validadas com bibliotecas XML.

  \item \emph{XML}: acrônimo para \emph{eXtensible Markup Language}, é uma especificação para uma linguagem de marcação de uso geral, permitindo que possam ser criados novas linguagens.

  \item \emph{XviD}: formato de compactação de vídeo competidor direto do DivX. Enquanto o DivX é um formato proprietário, o XVid é livre e de código aberto, disponível para diferentes plataformas.

  \item \emph{YouTube}: é um site na Internet que permite que seus usuários carreguem, assistam e compartilhem vídeos em formato digital. Foi fundado em fevereiro de 2005 por três pioneiros do PayPal, um famoso site da internet ligado a gerenciamento de doações. Foi comprada em 9 de Outubro de 2006 pelo Google, pela quantia de US\$1,65 bilhões em ações.

\end{itemize}
% chapter glossário (end)



% anexos (opcional)
%\addcontentsline{toc}{chapter}{\appendixname}
%\appendix
%

%


% indice remissivo (opcional)

\end{document}
