\chapter{CONSIDERAÇÕES FINAIS} % (fold)
\label{cha:consideracoes_finais}

\section{Trabalhos Futuros}
\label{sec:trabalho_futuros}

 Nesta seção serão descritos possíveis trabalhos futuros relacionados ao que foi realizado neste projeto. Alguns deles não puderam ser implementados devido às restrições de tempo ou por mudança no escopo do projeto.

\subsection{Web Semântica} % (fold)
\label{sub:web_semantica}

 A World Wide Web é um sistema de documentos em hipermídia que são interligados e executados na Internet. A maioria do conteúdo disponível na Web atualmente é projetado para a leitura por seres humanos. A utilização de programas de computador para interpretação desse conteúdo é complexa devido à baixa flexibilidade dos programas em relação aos humanos. Se por um lado os aplicativos são menos flexíveis que os seres humanos para interpretação de textos, por outro eles são capazes de desenvolver e analisar estruturas de dados complexas. A Web Semântica baseia-se nessa virtude do software. 

 Segundo Berners-Lee \cite{bemerslee2001sw}, a Web Semântica não é uma Web separada, mas um extensão da atual, na qual a informação é disponibilizada com sentido bem definido, aprimorando a capacidade de cooperação entre pessoas e computadores. Os documentos nesta extensão da Web são convenientes para utilização tanto por humanos como por programas.

 Da mesma forma que os humanos buscam informações em bases de dados descentralizadas na Web, com a adoção da Web Semântica agentes computacionais poderão fazer o mesmo. Essa descentralização de documentos permite que inconsistências ocorram, porém possibilita um rápido crescimento no volume de dados.

 A W3C realiza trabalhos para aprimorar, extender e padronizar a Web. Com o apoio de uma equipe especializada do consórcio, tecnologias para representação estrutural e semântica dos recursos na Web foram desenvoldidas, resultando em um conjunto de especificações para a Web Semântica. Atualmente este conjunto é basicamente composto pelo Resource Description Framework (RDF), a linguagem RDF Schema e a Web Ontology Language (OWL).

 Os mecanismos de busca atualmente utilizados na Web são capazes de listar os sites que contenham os termos desejados, bem como ordená-los de acordo com critérios de relevância. Considerando as dificuldades impostas à interpretação do conteúdo pelos computadores, fica a cargo do usuário a tarefa de identificar quais dos sites obtidos são coerentes com as expectativas de contexto e necessidade.

 Cientes disso, o grupo tinha proposto inicialmente um sistema de recomendação baseado em confiança com padrões de confiança. Todas as informações contidas no repositório seriam documentos semanticamente estruturados em padrões abertos. Com isso, todas as informações sobre avaliações e recomendações de produtos, além de perfis de usuários estariam disponíveis para importação por outros sistemas que entendem a linguagem adotada nesses documentos. Desse modo, a filtragem e localização de produtos mais relevantes aos usuários ganharia a poderosa ferramenta do processamento computacional.

% subsection web_semantica (end)

\subsection{Confiança Dependente do Tempo} % (fold)
\label{sub:confianca_dependente_do_tempo}

 Em \cite{sabater2001regret} o modelo de reputação proposto é baseado no cálculo do parâmetro de reputação a partir da somatórias dos pesos de impressões que o usuário tem de determinada entidade. Porém, o artigo levanta um ponto importante e crucial no modelo: a reputação ser dependente do tempo. Impressões antigas não podem ser consideradas com o mesmo peso das mais recentes. Para isso, os autores propõem a utilização de uma função dependente do tempo tal como a Equação~\ref{eq:REGRET}.
 
\begin{equation}
 R^t(IDB_a) = {\sum_{t_i\in{IDB_a}}}\rho(t,t_i)\times{W_i}
 \label{eq:REGRET} 
\end{equation}

 Onde $R^t(IDB_a)$ significa a reputação no tempo $t$ que o agente $a$ tem da impressão $IDB$ \textit{Impressions Database}. Os pesos das impressões são os $W_i$, que são multiplicados por $\rho(t,t_i)$ e somados para todas as impressões que o agente tem. A função $\rho(t,t_i)$ é uma função dependente do tempo, como visto na Equação~\ref{eq:rho_t}.
 
\begin{equation}
 \rho(t,t_i) = \frac{f(t_i,t)}{{\sum_{t_j\in{IDB_a}}}f(t_j,t)}
 \label{eq:rho_t} 
\end{equation}

 A função $f(t_i,t)$ dá valores próximos de t, podendo ser a função $f(t_i,t) = \frac{t_i}{t}$. Com isso, impressões antigas não têm o mesmo peso que impressões mais recentes.

 Considerando o nosso sistema proposto, o cálculo da confiança entre dois usuários poderia levar em conta o tempo em que as avaliações dos produtos recomendados foram feitas. Isso porque o usuário pode avaliar mau um produto sem conhecê-lo, apenas demonstrando o seu baixo interesse nele. Após alguns meses, o usuário pode vir a conhecer o produto e passar a ter interesse no mesmo. Com isso, a sua avaliação antiga não estaria condizente com o seu gosto atual.
 
 O grupo não realizou a implementação desse modelo de confiança devido às restrições de tempo para a realização do experimento.

% subsection confianca_dependente_do_tempo (end)

\subsection{Feedback das Recomendações} % (fold)
\label{sub:feedback_das_recomendacoes}

 O conteúdo de uma rede social é tão bom quanto a qualidade da informação compartilhada pelos seus usuários. Novos usuários muitas vezes não entendem o valor dessa contribuição e entram em uma rede apenas para observar os outros e não inserir nenhuma informação ou opinião sua de qualidade. Por isso, a motivação para que novos entrantes ajudem na elaboração do conteúdo da rede social é muito importante e um dos meios para que isso ocorra é o \textit{feedback} que o usuário ganha ao contribuir com algo\cite{burke2009fmm}.
 Desse modo, quando um usuário compartilha alguma informação ele quer saber o que os outros presentes na rede acharam. Esse \textit{feedback} motiva os usuários a compartilharem cada vez mais e, com isso, aumentar o conteúdo presente na rede.
 A proposta de trabalho futuro envolve a adição de \textit{feedbacks} aos usuários que avaliarem ou recomendarem produtos na rede social. O sistema pode mostrar quais amigos do usuário avaliaram o mesmo produto ou simplesmente mostrar a avaliação que uma pessoa deu a um produto recomendado pelo usuário. De acordo com\cite{burke2009fmm}, essa informação de retorno que o sistema dá a um usuário que contribui na rede motiva o mesmo a continuar compartilhando informações.
 Esses mecanismos de \textit{feedback} não foram implementados no sistema utilizado no experimento para que os participantes não tivessem nenhuma influência na hora de avaliar ou recomendar os produtos. Caso o participante recebesse informações de retorno do sistema com relação às suas ações no experimento, ele poderia tentar mudar a sua estratégia de recomendação ou avaliação de produtos e, desse modo, não se comportar na sua maneira natural de agir.

% subsection feedback_das_recomendacoes (end)

%section trabalhos futuros (end)

\section{Conclusão}
\label{conclusao}

% section conclusao (end)

% chapter trabalhos_futuros (end)