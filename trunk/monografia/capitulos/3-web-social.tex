\chapter{WEB SOCIAL} % (fold)
\label{cha:web_social}

\section{O que é Web Social}

O conceito de Web Social está relacionado diretamente com o termo Web 2.0, termo criado em 2004 pela empresa O'Reilly Medial, o que designa o uso da Internet como plataforma para novos tipos de comunidades e serviços. Antes do advento da Web2.0, o fluxo de informação era praticamente unidirecional, sendo que os dados eram disponibilizados pelo proprietário do site da internet e apenas visualizado pelas pessoas que o frequentavam. A troca de informações era muito baixa, dado que as pessoas não podiam comentar sobre o conteúdo a elas exposto, dando a sua opinião positiva ou negativa.

Com o surgimento da Web 2.0, a contribuição das pessoas tornou-se fundamental para a disponibilidade do conteúdo presente na Internet. O fluxo de informações vem se tornando bidirecional, ou seja, as pessoas visualizam um conteúdo presente em um site e podem comentar, alterar ou até mesmo adicionar novos dados.

Neste cenário, as pessoas se cadastram no site e criam perfis com as suas informações pessoais. Há também a possibilidade da criação de comunidades, com grande importância, pois o objetivo das pessoas que se cadastram em uma rede social é a de encontrar outras, amigas ou não, com os mesmos interesses. Isso facilita o seu engajamento e a criação de um conteúdo mais elaborado, já conhecido por todos no contexto da comunidade.

% Motivando novos usuários
\section{Qualidade da Informação}

% Adicionar referência para coldstart users
Para a formação do seu conteúdo, as redes sociais precisam das informações vindas dos seus usuários. Quanto mais as pessoas contribuem, mais rica será a rede social \cite{burke2009fmm}. A dependência dos usuários traz a necessidade de mostrar a eles o porquê de eles contribuirem. Uma rede social deve motivar os seus usuários a participar ativamente com as suas opiniões ou novas informações a serem disponibilizadas, mesmo sem receber em troca algum tipo de recompensa além da exposição das suas idéias. O desafio é mostrar aos novos usuários, chamados de \textit{coldstart users}, o motivo para eles contribuirem com o fluxo de informações.

%Em \cite{burke2009fmm} há um estudo específico relacionado às ações de novos usuários em uma grande rede social, o FaceBook\footnote{www.facebook.com}.

Durante a formação de uma rede social, o princípio básico adotado é sempre o cadastro inicial de algumas pessoas influentes, para que estas depois possam enviar convites para seus amigos e conhecidos. Também há a opção de convidar amigos presentes em outras redes, o que torna mais fácil o processo de localização de pessoas conhecidas. Após isso, existem técnicas adotadas para motivar os usuários a contribuirem com dados. Segundo \cite{burke2009fmm}, uma das técnicas adotadas é mostrar aos usuários o que seus amigos estão fazendo na rede. Essa técnica tem como base o comportamento humano de, ao chegar em um novo local com pessoas já interagindo entre si, primeiro observar para depois agir. Sabendo o comportamento dos seus amigos, os novos usuários agirão de maneira semelhante \cite{burke2009fmm}.

Mesmo fazendo com que as pessoas compartilhem informações, a qualidade destas ainda depende do propósito da rede. Algumas redes sociais visam apenas diversão, como o Orkut\footnote{www.orkut.com} e o MySpace\footnote{www.myspace.com}, porém existem redes sociais voltadas para a criação de perfis profissionais, sendo o LinkedIn\footnote{www.linkedin.com} um exemplo. As informações contidas no Orkut e no MySpace não são necessariamente de alta qualidade, uma vez que muitas pessoas não utilizam sua identificação real. Já no LinkedIn, a maioria das pessoas entram com suas informações reais, fazendo com que a qualidade desses dados seja muito mais relevante.

% Por que ela depende de recomendação?
%\section{Necessidade de Sistemas de Recomendação}



% chapter web_social (end)

% Tecnologias e conceitos empregados, contextualização do Projeto de Formatura em sua área de aplicação, revisão da literatura.
