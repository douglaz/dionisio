\chapter{ANÁLISES E RESULTADOS} % (fold)
\label{cha:analises_e_resultados}

\section{Introdução}
\label{cha:introducao}

 A análise dos resultados consistiu na elaboração de gráficos e tabelas com os dados retirados da base do experimento realizado. Os seguintes dados foram considerados para a análise:
 
\begin{itemize}
	\item Grupos de usuários
	
	\subitem Para se ter noção dos laços de amizades entre os participantes cadastrados no experimento. São 12 grupos no total, sendo que cada um deles possui no máximo 5 pessoas.
	
	\item Recomendações realizadas por amigos
	
	\subitem Informações dos produtos que foram recomendados na etapa 3 do experimento, onde os participantes deveriam recomendar 5 produtos a cada amigo presente no seu grupo. Cada participante possui no máximo 20 recomendações, sendo que alguns possuem apenas 15 devido à desistência de participantes do seu grupo.
	
	\item Recomendações realizadas por desconhecidos
	
	\subitem Informações dos produtos que foram recomendados na etapa 4 do experimento, onde os participantes deveriam recomendar 1 produto a alguns participantes de diferentes grupos, considerados desconhecidos. Cada participante possui no máximo 10 recomendações realizadas por desconhecidos, sendo que alguns possuem menos de 10 recomendações devido à desistência de participantes no experimento.
	
	\item Recomendações realizadas pelo sistema
	
	\subitem Todas as recomendações que o sistema realizou para os participantes. Contém a informação de qual participante recebeu a recomendação e qual foi o produto recomendado.
	
	\item Avaliação prevista do produto pelo sistema
	
	\subitem Ao recomendar um produto a um participante, o sistema calcula uma nota prevista para o mesmo. Essas informações foram armazenadas e consideradas durante a análise e exposição dos dados do experimento.
	
	\item Avaliações dos produtos
	
	\subitem Avaliações de todos as avaliações de produtos no sistema. Contém os produtos avaliados na duas primeiras etapas do experimento, quando os participantes avaliaram 20 produtos em comum e 10 produtos de seu interesse, e as avaliações de produtos recomendados tanto pelos participantes como pelo sistema.
	
	\item Algoritmo utilizado para a recomendação
	
	\subitem Dentre as recomendações realizadas pelo sistema, foi retirada da base de dados a informação de qual algoritmo foi utilizado para gerar a recomendação.
	
\end{itemize}

 De posse dos dados, foram feitas as seguintes análises:
 
\begin{itemize}
	\item Análise Comparativa dos Algoritmos de Recomendação
	\item Análise de Rejeição das Recomendações
	\item Taxa de Serendipidade
	\item Análise do Algoritmo Baseado em Confiança
\end{itemize}

 Cada análise será discutida nas seções a seguir.
 
\section{Análise Comparativa dos Algoritmos de Recomendação}
\label{sec:analise_comparativa_dos_algoritmos_de_recomendacao}

 
 
\subsection{Recomendações do Sistema}

\subsection{Recomendações Diretas}


% section analise_comparativa_dos_algoritmos_de_recomendacao (end)

\section{Análise de Rejeição das Recomendações}
\label{sec:analise_de_rejeicao_das_recomendacoes}

\subsection{Recomendações do Sistema}

\subsection{Recomendações Diretas}

% section analise_de_rejeicao_das_recomendacoes (end)

\section{Taxa de Serendipidade}
\label{sec:taxa_de_serendipidade}

% section taxa_de_serendipidade (end)

\section{Análise do Algoritmo Baseado em Confiança}
\label{sec:analise_do_algoritmo_baseado_em_confianca}

% section analise_do_algoritmo_baseado_em_confianca (end)

\section{Análise de Desempenho}
\label{sec:analise_de_desempenho}
% TODO falar sobre os problemas enfrentados com as consultas SQL e com as estrelinhas para a avaliação do produto (javascript).

% section analise_de_desempenho (end)