% The Digital TV has appeared as the new paradigm for interaction between viewers and TV shows. Its development in the country also represents the great effort the Brazilian government has shown for digital inclusion. The aim of this project is to specify and develop a portable device for Digital TV signal capturing through the USB protocol. The technical requirements and project specifications are detailed through this paper, which includes the complete description of the necessary steps for developing the device. It is based on an USB development kit, which counts on an FPGA for dealing and processing of the digital signals. The device can be used on several operating systems, which guarantees the portability desired for this project. The main results are shown through example applications that use the device as main source for displaying and recording video streams, delivered by a transport stream generator device. This paper also describes the use of the device with the Brazilian Digital TV standard (SBTVD), besides analyzing different kinds of interactivity possible to exist on Digital TV applications running over the Brazilian middleware Ginga. In conclusion, this paper presents and discusses new applications that may use an existing connection to the Internet as the return channel, bringing the services already available on the Internet to the Digital TV world and expanding their functional capabilities and visual appeal.
% 
% Keywords: Engineering. Computer Engineering. Digital TV. USB. Return Channel.