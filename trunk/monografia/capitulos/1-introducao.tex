\chapter{INTRODUÇÃO} \pagenumbering{arabic} % (fold)

\section{Motivação} % (fold)
\label{sec:motivação}

% O que é importante para: nós, comunidade científica, usuários, mercado.

% Apresentar a motivação e justificativa para a realização do trabalho (por exemplo, sua aplicabilidade prática, comparação com alternativas já existentes, potencial de  aprendizado e evolução, etc).

% http://www.uxmatters.com/mt/archives/2009/03/including-recommendations-in-user-interfaces-to-enhance-motivation.phps

 Sistemas de Recomendação sugerem itens que as pessoas possam gostar, baseado no comportamento prévio delas. Fazendo suposições pertinentes sobre o tipo de objetos em que elas estão interessadas, é possível conquistar a sua confiança. A vantagem para os usuários é a facilidade de encontrar a informação, sem ter a árdua tarefa de procurá-la.

 As redes sociais online têm modificado a forma com que as empresas utilizam a comunicação para o comércio. Pessoas estão utilizando a Web para encontrar outras pessoas com interesses similares, fazer compras de forma mais eficiente, aprender sobre produtos e serviços e reclamar sobre produtos malfeitos\cite{marketing_social_web}.

 A Web está rapidamente se tornando a mídia mais importante para o marketing. A tendência é que as pessoas cada vez mais bloqueiem os anúncios indesejados e queiram ter a capacidade de encontrar os produtos relevantes no momento adequado. É nesse contexto que surge a necessidade de uma plataforma que facilite a colaboração e que permita a criação e classificação de conteúdo pelos consumidores, de forma a permitir uma escolha mais inteligente dos melhores produtos e serviços, e ao mesmo tempo criando uma mecanismo de feedback para as empresas interessadas.

 Devido à grande variedade atual de produtos e serviços, as pessoas têm cada vez mais dificuldade nas suas escolhas e na argumentação sobre a possível decisão. Quanto maior for a quantidade de produtos similares de fabricantes diferentes, mais as pessoas se vêem desnorteadas e sem saber se a decisão realizada foi a mais correta.

% paragraph paragraph_name (end)
% section motivação (end)


\section{Objetivos} % (fold)
\label{sec:objetivos}
% Apresentar de forma precisa e concisa o objetivo do projeto.

 O principal objetivo deste projeto é criar um sistema de recomendação baseado em recursos disponíveis na Web, que possibilite a sugestão de itens confiáveis e relevantes ao usuário.

% Talvez comentar o experimento

% section objetivo (end)

\section{Metodologia} % (fold)
\label{sec:metodologia}

 % Comentar sobre as reuniões que nós fazíamos para definir o foco do projeto e as próximas atividades.
 A metodologia utilizada pelo grupo foi baseada em reuniões com os orientadores onde eram discutidos os pontos principais do projeto. As reuniões eram inicialmente agendadas de acordo com as entregas de documentação exigidas pela disciplina, e outras reuniões, quando necessárias, podiam ser marcadas por e-mail ou pessoalmente para o esclarecimento de questões relacionadas ao foco e aos objetivos do projeto. Todo o desenrolar do projeto foi acompanhado pelos orientadores que opinavam sobre os argumentos dos orientados e indicavam possíveis mudanças e redirecionamentos.

 % Falar sobre a realização de um experimento.
 Foi realizado um experimento prático com a participação de pessoas externas ao projeto para o colhimento de resultados da utilização do sistema proposto no projeto.

 % Falar sobre os critérios utilizados para montar a base de dados de produtos.
 Para a formação da base de dados que comporia o cadastro de produtos utilizado no experimento, foram escolhidos os seguintes aspectos:

\begin{itemize}
	\item Diversidade de produtos
	\subitem Foram consideradas diversas categorias de produtos para que houvesse uma heterogeneidade no cadastro.
	\item Produtos com alto índice de relevância nas categorias
	\subitem Os produtos mais relevantes em cada categoria escolhida foram cadastrados com o intuito da base de dados não ser composta apenas de produtos desconhecidas aos participantes do experimento.
\end{itemize}

 % Falar sobre TDD (Test Driven Development).
 % TODO
 A técnida de desenvolvimento de software utilizada foi o TDD\footnote{Do inglês \textit{Test-driven Development.} que se baseia na realização de pequenos ciclos de desenvolvimento. Inicialmente são escritos alguns casos de teste para as novas funcionalidades ou melhorias.

 % Desenvolvimento das etapas durante o experimento.
 O experimento foi dividido em etapas, sendo que enquanto uma etapa não era concluída por todos os participantes, as etapas subsequentes não ficavam disponíveis. Com isso, o desenvolvimento das funcionalidades disponíveis pelo sistema foi gradual, proporcionando a correção de erros nas etapas a serem ainda liberadas.

 Além disso, enquanto o experimento estava no ar, os participantes que já haviam terminado determinada etapa comunicavam espontaneamente por e-mail as dificuldades ou facilidades que tiveram durante a realização da mesma. Isso facilitava a correção imediata de erros com o sistema no ar e a mitigação dos mesmos na etapas seguintes.

 % Controle de versão SVN
 O controle de versão do código criado foi feito com o SVN\footnote{Do inglês \textit{Subversion}. SVN é um sistema de controle de versão de documentos quaisquer.}. Esse sistema de controle ajudou muito no desenvolvimento paralelo do sistema do projeto por todos os integrantes do grupo. Os arquivos utilizados durante todo o desenvolvimento do projeto também foram versionados com o SVN.

% section metodologia (end)

\section{Organização} % (fold)
%\label{sec:organização}
% Apresentar a organização do documento: o que cada capítulo, anexo e apêndice aborda.


% section organização (end)

% chapter web_social (end)