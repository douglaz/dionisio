\chapter{INTRODUÇÃO} \pagenumbering{arabic} % (fold)

\section{Objetivo} % (fold)
\label{sec:objetivo}
% Apresentar de forma precisa e concisa o objetivo do projeto.

 O principal objetivo deste projeto é criar um sistema de recomendação baseado em recursos disponíveis na Web, que possibilite a sugestão de itens confiáveis e relevantes ao usuário.

% section objetivo (end)

\section{Motivação} % (fold)
\label{sec:motivação}

% O que é importante para: nós, comunidade científica, usuários, mercado.

% Apresentar a motivação e justificativa para a realização do trabalho (por exemplo, sua aplicabilidade prática, comparação com alternativas já existentes, potencial de  aprendizado e evolução, etc).

% http://www.uxmatters.com/mt/archives/2009/03/including-recommendations-in-user-interfaces-to-enhance-motivation.phps

 Sistemas de Recomendação sugerem itens que as pessoas possam gostar, baseado no comportamento prévio delas. Fazendo suposições pertinentes sobre o tipo de objetos em que elas estão interessadas, é possível conquistar a sua confiança. A vantagem para os usuários é a facilidade de encontrar a informação, sem ter a árdua tarefa de procurá-la.

 As redes sociais online têm modificado a forma com que as empresas utilizam a comunicação para o comércio. Pessoas estão utilizando a Web para encontrar outras pessoas com interesses similares, fazer compras de forma mais eficiente, aprender sobre produtos e serviços e reclamar sobre produtos malfeitos\cite{marketing_social_web}.

 A Web está rapidamente se tornando a mídia mais importante para o marketing. A tendência é que as pessoas cada vez mais bloqueiem os anúncios indesejados e queiram ter a capacidade de encontrar os produtos relevantes no momento adequado. É nesse contexto que surge a necessidade de uma plataforma que facilite a colaboração e que permita a criação e classificação de conteúdo pelos consumidores, de forma a permitir uma escolha mais inteligente dos melhores produtos e serviços, e ao mesmo tempo criando uma mecanismo de feedback para as empresas interessadas.

 Devido à grande variedade atual de produtos e serviços, as pessoas têm cada vez mais dificuldade nas suas escolhas e na argumentação sobre a possível decisão. Quanto maior for a quantidade de produtos similares de fabricantes diferentes, mais as pessoas se vêem desnorteadas e sem saber se a decisão realizada foi a mais correta.
 
 % TODO: Escrever sobre Web Semântica

% paragraph paragraph_name (end)
% section motivação (end)

%\section{Organização} % (fold)
%\label{sec:organização}

% Não fazer ainda!!!
% Apresentar a organização do documento: o que cada capítulo, anexo e apêndice aborda.


% section organização (end)

% chapter web_social (end)