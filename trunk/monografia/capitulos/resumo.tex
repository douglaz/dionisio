O comércio eletrônico brasileiro tem evoluído em termos de serviços prestados ao consumidor. Este trabalho apresenta a especificação e implementação de um sistema online de recomendação para facilitar o encontro do usuário com os produtos e serviços mais relevantes a ele. Para isso, são descritos em detalhes os requisitos funcionais e não funcionais do sistema, os algoritmos de recomendação utilizados e as escolhas na implementação do projeto. 
O sistema projetado é composto por três algoritmos de recomendação, sendo dois deles já difundidos, enquanto o terceiro é um novo, e que foi proposto para levar em consideração as relações de confiança do usuário com seus pares. 
A execução deste trabalho incluiu a realização de um experimento com pessoas interagindo com o sistema de recomendação através de uma rede social. A análise do experimento é apresentada através de gráficos e dados estatísticos que comparam os algoritmos utilizados. 
Os resultados do projeto podem ser aplicados no futuro no desenvolvimento de lojas virtuais online e/ou aplicativos de redes sociais.