% A TV digital desempenhará papel importante no processo de inclusão digital previsto pelo Governo Brasileiro. Com o intuito de contribuir para o desenvolvimento da mesma no país, este trabalho apresenta a especificação e implementação de um dispositivo de captura de TV digital via USB. São detalhados os requisitos técnicos e escolhas de projeto realizadas para a implementação do dispositivo, que foi baseado em um kit de desenvolvimento USB com uma FPGA como unidade de tratamento e processamento de sinais. O dispositivo opera em diferentes sistemas operacionais, uma vez que a portabilidade é um dos objetivos deste trabalho. Os resultados são apresentados através de aplicações que realizam a reprodução dos vídeos no computador e também sua gravação em dispositivos de armazenamento. O trabalho aborda também a integração do dispositivo ao padrão brasileiro de TV Digital, além de analisar e caracterizar os diferentes tipos de aplicações e níveis de interatividade possíveis de serem utilizados com o middleware brasileiro Ginga. Por fim, são discutidas e propostas algumas aplicações interativas que utilizam a conexão normal à Internet como canal de interatividade, expandindo assim a usabilidade dos serviços disponíveis na Internet para as aplicações de TV digital, agregando maior valor às aplicações interativas possíveis de serem desenvolvidas.
% 
% Palavras-chave: Engenharia. Engenharia da Computação. TV Digital. USB. Canal de Retorno.