\chapter{INTRODUÇÃO}
\pagenumbering{arabic}

\section{Objetivos}

     O objetivo principal deste projeto é criar uma plataforma social online que estimule colaboração entre os usuários e possibilite a identificação das opiniões e expectativas a respeito de produtos e/ou serviços.



\section{Motivação}

 As plataformas sociais online têm modificado a forma com que as empresas utilizam a comunicação para o comércio. Pessoas estão utilizando a Internet para encontrar outras pessoas com interesses similares, fazer compras de forma mais eficiente, aprender sobre produtos e serviços e reclamar sobre produtos malfeitos e serviços pobres\cite{marketing_social_web}.
 
    A internet está rapidamente se tornando a mídia mais importante para o marketing. Os primeiros publicitários a utilizarem esse meio, que enxergaram essa mídia apenas como outro canal, não aproveitaram a melhor forma de fazer marketing na internet. A tendência é que as pessoas, cada vez mais bloqueiem os anúncios indesejados e queiram ter a capacidade de encontrar os produtos relevantes no momento adequado. É nesse contexto que surge a necessidade de uma plataforma que facilite a colaboração e que permita a criação e classificação de conteúdo pelos consumidores, de forma a permitir uma escolha mais inteligente dos melhores produtos e serviços e ao mesmo tempo criando uma mecanismo de feedback para as empresas interessadas.
 
    Devido à grande variedade atual de produtos e serviços, as pessoas têm cada vez mais dificuldade nas suas escolhas e na argumentação sobre a possível decisão. Quanto mais produtos similares de fabricantes diferentes ou serviços realizados por diversos prestadores, mais as pessoas se vêem desnorteadas e sem saber se a decisão realizada foi a mais correta. O mesmo ocorre com as empresas na hora de lançar os seus produtos. Os seus clientes pedem cada vez mais inovações e não há como saber a opinião de cada pessoa ou de um determinado grupo de pessoas sobre as novas características agregadas a um produto já lançado ou sobre novos produtos que agregarão tais inovações tecnológicas. Se a empresa tivesse posse dessas informações, ela também poderia decidir sobre o seu melhor caminho, ou seja, em quais tecnologias ou tipos de serviço investir.
 
    O projeto visa ter como resultado um produto que auxilia esse processo de decisão com base na opinião das pessoas diretamente envolvidas. O sistema não pretende influenciar as pessoas e nem ao menos estreitar as opções que elas têm, mas sim organizar as opiniões de todos e deixar transparecer a outros a opinião alheia com o intuito de fazer com que os que deixaram a sua opinião possam ser ouvidos pelos outros e ter a sua escolha levada em conta pelo sistema.

\section{Alterações no Escopo do Projeto}

    Inicialmente o escopo do projeto englobava a criação de uma rede social utilizada como fim para recomendações de bares e restaurantes. Os usuários procuravam estabelecimentos cadastrados na rede e conseguiam informações sobre os serviços prestados de acordo com notas e comentários feitos por outros usuários. Com isso, eles podiam recomendar bares e restaurantes para outros usuários e também para tags. As tags serviam para classificar estabelecimentos em determinados grupos para facilitar a busca e identificação dos mesmos.
 
    Também dentro do escopo estavam definidos os donos dos estabelecimentos que podiam tomar o cadastro próprio e gerenciá-lo para obter as informações estatísticas de quais tipos de pessoas haviam frequentado o seu bar ou restaurante. Com isso, formava-se um canal de comunicação com o cliente que consistia na resposta a perguntas feitas diretamente ao estabelecimento, além de divulgação de informações e novidades sobre o seu negócio.
 
    As alterações no escopo consistem principalmente no uso da rede social que não será mais utilizada como "fim" e sim como "meio". O projeto agora visa a criação de uma ferramenta de groupware para que os usuários que utilizam a rede possam marcar encontros e reuniões nos bares e restaurantes cadastrados. Groupware é um software colaborativo que apóia o trabalho em grupo.

    Um usuário pode propor aos seus amigos, por meio da rede social, um encontro em um determinado estabelecimento, sendo que os convidados podem aceitar ir ou não ao evento na data e horário estipulados. No caso negativo, a pessoa pode sugerir um outro evento ou modificações no evento atual. No final, o sistema transparece a opinião mais adequada para o grupo considerando as restrições de cada um dos integrantes.
 
\subsection{Aplicações do Produto}

    O produto pode ser aplicado na organização de amigos que queiram se encontrar em um bar ou restaurante e em empresas que queiram ouvir o que seus clientes estão falando sobre o seu produto. Além disso, pode ser usado em academias que queiram mudar a grade horária de suas atividades e aulas, assim como escolas de inglês e cursos de informática. Em todo e qualquer lugar em que a opinião de cada pessoa envolvida seja importante o produto pode ser aplicado. 

    No caso da organização de amigos que queiram se encontrar em um bar ou restaurante, o sistema funcionaria inicialmente com a recomendação de uma pessoa para um grupo de amigos propondo um encontro em um determinado estabelecimento em uma data e horário. O aplicativo deve ouvir a todos os que foram relacionados expondo a sua opinião sobre a proposta e informando se a pessoa concorda em estar no local e hora escolhidos. Qualquer um deles pode propor outro tipo de bar ou restaurante caso não goste do estabelecimento em questão e, desse modo, as pessoas poderão também decidir sobre essa nova proposta. Após todos terem opinado ou um tempo limite ter sido atingido (por exemplo: um dia antes da data marcada), o sistema indica, também podendo ser de acordo com o respaldo e poder de cada um dos envolvidos, qual foi a decisão da maioria. Quando utilizado como ferramenta de escuta das opiniões de clientes e consumidores, o aplicativo pode funcionar como canal de comunicação da diretoria de academias ou escolas de inglês. Desse modo, a diretoria propõe novas aulas, cursos ou grades horárias e obtém do sistema a opinião dos envolvidos (alunos neste caso). Com uma resposta de maioria positiva, a diretoria teria um embasamento na sua tomada de decisão. Em empresas que lancem produtos com inovações tecnológicas, há a possibilidade da empresa escutar aos seus clientes e verificar se um dado produto tem aceitação pela maioria interessada ou não.
