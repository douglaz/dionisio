\chapter*{Glossário} % (fold)
\label{cha:glossario}

\begin{itemize}

  \item \emph{ADSL}: acrônimo de \emph{Asymmetric Digital Subscriber Line}, é uma forma de transmissão de dados de alta velocidade utilizando linhas telefônicas comuns, em freqüências maiores que os seres humanos conseguem escutar.

  \item \emph{BNC}: é um tipo de conector cujo nome vem de seus criadores: ``bayonet Neil-Concelman''. Sua grande característica é o sistema de trava, tipo twist-lock (gira e trava), que possibilita grande segurança na conexão. É bastante utilizado nos equipamentos profissionais de vídeo.

  \item \emph{Broadcast}: é um modo de difusão de sinais em que é transmitido o mesmo conteúdo para todos os receptores. Numa transmissão de TV, por exemplo, todas as pessoas sintonizadas no mesmo canal assistem ao mesmo programa. Em Internet, o termo é usado muitas vezes para designar o envio de uma mensagem par todos os membros de um grupo, em vez da remessa para membros específicos.

  \item \emph{Buffer}: é uma área de armazenamento que compensa diferentes velocidades de fluxos de dados ou temporizações de eventos, ao transferir dados de um dispositivo para outro.

  \item \emph{CD}: acrônimo de Compact Disc, é um padrão de armazenamento óptico para dados digitais.
  Conversor A/D: um Conversor Analógico-Digital é componente de um sistema responsável por converter dados analógicos para digitais através da amostragem de um sinal contínuo e sua posterior discretização gerando valores numéricos digitais.

  \item \emph{Classpath}: é um argumento passado para a Java Virtual Machine indicando onde procurar classes e pacotes para carregamento dinâmico.

  \item \emph{CPqD}: {C}entro de {P}es{q}uisa e {D}esenvolvimento em Telecomunicações.

  \item \emph{DivX}: é um formato de compactação de vídeo criado pela DivxNetworks Inc.

  \item \emph{Driver}: um driver é um componente de software responsável por estabelecer a comunicação entre hardware e software, provendo comandos para enviar e receber dados de um dispositivo instalado.

  \item \emph{DVD}: acrônimo de Digital Versatile Disc, é a geração seguinte ao CD, possibilitando um armazenamento maior de dados.

  \item \emph{ECMAScript}: é uma linguagem de programação baseada em scripts, padronizada pela Ecma International na especificação ECMA-262. A linguagem é bastante usada em tecnologias para Internet, sendo esta base para a criação do JavaScript/JScript e também do ActionScript.

  \item \emph{FIFO}: acrônimo para {F}irst {I}n, {F}irst {O}ut (que em português significa primeiro a entrar, primeiro a sair) refere-se a estruturas de dados do tipo fila onde os elementos vão sendo colocados e retirados (ou processados) por ordem de chegada.

  \item \emph{GEM}: acrônimo de \emph{Globally Executable MHP} \cite{gem}, é uma parte do MHP independente dos padrões de transmissão europeus, criado com a finalidade de padronizar partes de todos os padrões de \emph{middleware} mundiais e possibilitar a criação de aplicações que funcionem em qualquer \emph{middleware} que seja compatível com o GEM.

  \item \emph{GSM}: acrônimo de Global System for Mobile Communications é um dos principais padrões para telefonia móvel existente.

  \item \emph{Heap}: Heap de objetos é o nome dado a parte da memória do computador que contém a estrutura de dados responsável por armazenar todos os objetos durante a execução da máquina virtual Java.

  \item \emph{HTTP}: acrônimo para {H}yperText {T}ransfer {P}rotocol (Protocolo de Transferência de Hipertexto), utilizado para transferência de dados na rede mundial de computadores, a World Wide Web.

  \item \emph{IPC (Inter-Process Communication)}: é o grupo de mecanismos que permite aos processos transferirem informação entre si. Entre estes mecanismos podem ser citados pipes, filas de mensagens e memória compartilhada.

  \item \emph{Java}: é uma linguagem de programação orientada a objeto desenvolvida na década de 90 pelo programador James Gosling, na empresa Sun Microsystems.

  \item \emph{JavaBeans}: são componentes reutilizáveis de software escritos em linguagem Java, e que seguem algumas convenções de modo a permitir que ferramentas possam utilizá-los e manipulá-los.

  \item \emph{JavaTV}: é uma biblioteca Java que contempla a maior parte dos recursos necessários para a operação de sistemas receptores de TV digital, simplificando assim o desenvolvimento de softwares, uma vez que os programadores de aplicativos podem se voltar ao tema principal da aplicação em desenvolvimento.

  \item \emph{JNI (Java Native Interface)}: é um padrão de programação que permite que a máquina virtual da linguagem Java acesse bibliotecas construídas com o código nativo de um sistema.

  \item \emph{Lista ligada}: é uma estrutura de dados linear e dinâmica composta por células que apontam para o próximo elemento da lista.

  \item \emph{Metodologia (Processo) de Desenvolvimento Ágil}: metodologias de desenvolvimento ágil foram pensadas de forma a minimizar riscos no desenvolvimento de software através períodos mais curtos de lançamento, chamados iterações, que tipicamente levam de uma quatro semanas. Métodos ágeis prezam mais a comunicação face-a-face que a documentação, como forma de acelerar o processo de desenvolvimento.

  \item \emph{MHP}: acrônimo de \emph{Multimedia Home Platform} \cite{mhp}, é um padrão aberto de \emph{middleware} para TV Digital, adotado principalmente pelos países europeus. Está incluso em outro padrão maior, o \emph{Digital Video Broadcasting} (DVB) \cite{dvb} que agrupa todas as características da tecnologia de TV Digital européia.

  \item \emph{Middleware} (conforme \cite{ginga}): é a camada de software intermediário que permite o desenvolvimento de aplicações interativas para a TV Digital de forma independente da plataforma de hardware dos fabricantes de terminais de acesso (\emph{set-top boxes}).

  \item \emph{MPEG}: acrônimo para {M}oving {P}icture {E}xperts {G}roup, é o grupo de trabalho da Organização Internacional para Padronização (ISO) para o desenvolvimento de padrões para vídeo e áudio digitais.

  \item \emph{MPEG2}: é um padrão de compressão e codificação de vídeo para difusão e comunicações, bem como para armazenamento em meios diversos, tais quais os ópticos.

  \item \emph{MPEG-TS}: MPEG {T}ransport {S}tream (TS, TP, ou MPEG-TS) é um protocolo de  comunicação para transmissão de áudio, vídeo e dados, especificado pelo padrão  ISO/IEC 13818-1. Permite multiplexar vídeo e áudio digital sincronizando a saída. Possui correção de erro e transporte, e é usado para difusão de aplicações como DVB e ATSC.

  \item \emph{MOV}: formato de vídeo criado pela Apple, é um container para imagem, diversas trilhas, efeitos e textos. Sua base foi aprovada pela ISO como padrão para MPEG4 Part. 14.

  \item \emph{Multiplexação no tempo}: a multiplexação no tempo consiste na transmissão de dois ou mais sinais ou fontes de bits simultaneamente através de um único canal pela divisão do tempo em pequenos compartimentos de tamanhos fixos, onde são transmitidos alternadamente um pouco de cada sinal.

  \item \emph{Overhead}: Em computação overhead é geralmente considerado qualquer processamento ou armazenamento em excesso, seja de tempo de computação, de memória, de largura de banda ou qualquer outro recurso que seja requerido para ser utilizado ou gasto para executar uma determinada tarefa.

  \item \emph{Pipe (UNIX)}: é o redirecionamento da saída padrão de um programa para a entrada padrão de outro.

  \item \emph{{S}et-{t}op {B}ox ({STB})}: é o termo que descreve um equipamento que se conecta a um televisor e a uma fonte externa de sinal, transformando este sinal em conteúdo no formato que possa ser apresentado em uma tela.

  \item \emph{SBTVD} \cite{sbtvd}: acrônimo para {S}istema {B}rasileiro de {TV} {D}igital.

  \item \emph{SMS}: acrônimo para {S}hort {M}essage {S}ervice. Tecnologia amplamente utilizada em telefonia celular para a transmissão de mensagens de texto curtas.

  \item \emph{SOAP}: acrônimo de \emph{Service Oriented Architecture Protocol}, é um protocolo de troca de mensagens em formato XML.

  \item \emph{Thread}: é uma forma de um processo dividir a si mesmo em duas ou mais tarefas que podem ser executadas simultaneamente.

  \item \emph{UML}: acrônimo de Unified Modelling Language, é um padrão gráfico de especificação para modelagem de objetos, e uma ferramenta importante no processo de desenvolvimento de software.

  \item \emph{{USB}}: acrônimo de {U}niversal {S}erial {B}us, é uma especificação para interfaces de comunicação serial de dados. É padronizado pelo {USB} {I}mplementers {F}orum ({USB-IF}), que possui membros como Apple Inc., Hewlett-Packard, NEC, Microsoft e Intel.

  \item \emph{W3C}: acrônimo de World Wide Web Consortium, é o principal órgão de padronização para a Web (Internet).

  \item \emph{Web Services}: é definido pelo W3C como um sistema de software projetado para dar suporte à comunicação interoperável entre duas máquinas utilizando uma rede. Essa comunicação é feita através de mensagens XML utilizando-se servidores web, em um padrão denominado SOAP.

  \item \emph{Wi-Fi}: é uma marca registrada pertencente à Wireless Ethernet Compatibility Alliance (WECA), e é a abreviatura de \emph{wireless fidelity}, sendo uma tecnologia de interconexão entre dispositivos sem fio, usando o protocolo IEEE 802.11.

  \item \emph{WiMax}: acrônimo de \emph{Worldwide Interoperability for Microwave Access}, é o nome comercial para o padrão IEEE 802.16, que especifica uma interface sem fio para redes metropolitanas, agregando conhecimentos e recursos mais recentes ao padrão Wi-Fi visando melhor performance de comunicação.

  \item \emph{WMV}: \emph{Windows Media Video} é o nome para uma série de formatos de vídeo compactados criados pela Microsoft.

  \item \emph{XHTML}: acrônimo para \emph{eXtensible HyperText Markup Language}, é uma linguagem de marcação com as mesmas marcas do HTML, porém com uma sintaxe mais rigorosa pois é baseada no XML, e dessa podem ser validadas com bibliotecas XML.

  \item \emph{XML}: acrônimo para \emph{eXtensible Markup Language}, é uma especificação para uma linguagem de marcação de uso geral, permitindo que possam ser criados novas linguagens.

  \item \emph{XviD}: formato de compactação de vídeo competidor direto do DivX. Enquanto o DivX é um formato proprietário, o XVid é livre e de código aberto, disponível para diferentes plataformas.

  \item \emph{YouTube}: é um site na Internet que permite que seus usuários carreguem, assistam e compartilhem vídeos em formato digital. Foi fundado em fevereiro de 2005 por três pioneiros do PayPal, um famoso site da internet ligado a gerenciamento de doações. Foi comprada em 9 de Outubro de 2006 pelo Google, pela quantia de US\$1,65 bilhões em ações.

\end{itemize}
% chapter glossário (end)

