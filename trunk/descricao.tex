\documentclass[a4paper,12pt]{article}

\usepackage[brazilian,english]{babel}
\usepackage{indentfirst}
\usepackage{a4wide}
\usepackage[section]{placeins}
\usepackage[utf8]{inputenc}
% \usepackage[Conny]{fncychap}
\usepackage[T1]{fontenc}
\usepackage{txfonts}
\usepackage{type1cm}
\usepackage{courier}
\usepackage[scaled]{helvet}
\usepackage{epigraph}
\renewcommand*\familydefault{\sfdefault}

\hyphenation{
  es-ta-be-le-ci-do
  a-de-qua-da-men-te
  pro-ble-mas
  di-men-sio-na-men-to
  mo-de-lo
}

% Controlar linhas orfas e viuvas
\clubpenalty=10000
\widowpenalty=10000
\displaywidowpenalty=10000

% Formatar notas de rodape
\usepackage[hang]{footmisc}
\setlength{\footnotemargin}{1em}

\usepackage{url}
%% Define a new 'leo' style for the package that will use a smaller font.
\makeatletter
\def\url@leostyle{%
  \@ifundefined{selectfont}{\def\UrlFont{\sf}}{\def\UrlFont{\small\ttfamily}}}
\makeatother
%% Now actually use the newly defined style.
\urlstyle{leo}

\usepackage[left=3cm,top=3cm,right=2cm,bottom=2cm,includehead,ignoremp]{geometry}

\usepackage[small]{titlesec}
% \titlespacing{\chapter}{0pt}{-50pt}{*20}[0pt]
\titlespacing{\chapter}{0pt}{*-10}{*5}

% Running Headers and footers
\usepackage{fancyhdr}
% \pagestyle{fancy}
\pagestyle{headings}
% Redefine plain page style
\fancypagestyle{plain}{
  \fancyhf{}
  \renewcommand{\headrulewidth}{0pt}
  \fancyhead[LE,RO]{\thepage}
}

\linespread{1.3}
%\pdfpagewidth=\paperwidth
%\pdfpageheight=\paperheight

% Para habilitar cálculos em dimensões
\usepackage{calc}

% Multipart figures
%\usepackage{subfigure}

% More symbols
%\usepackage{amsmath}
%\usepackage{amssymb}
%\usepackage{latexsym}

% Surround parts of graphics with box
\usepackage{boxedminipage}

% Package for including code in the document
\usepackage{listings}

% If you want to generate a toc for each chapter (use with book)
%\usepackage{minitoc}

% This is now the recommended way for checking for PDFLaTeX:
\usepackage{ifpdf}

\ifpdf
\usepackage[pdftex]{graphicx}
\else
\usepackage{graphicx}
\fi

\title{Dionisio}
\author{Allan Douglas R. de Oliveira \\ allandouglas@gmail.com \and
Leonardo Nicacio Bessa \\ leobessa@gmail.com \and
Thiago Rodrigues Andrade \\ suffragium@gmail.com}
\date{2009-09}

\usepackage[absolute]{textpos}
\begin{document}

\ifpdf
\DeclareGraphicsExtensions{.png, .pdf, .jpg, .tif}
\else
\DeclareGraphicsExtensions{.eps, .jpg}
\fi

%\maketitle

\pagestyle{empty}
\section{Definições}

\paragraph{Confiança:} grau atribuído por um usuário para outro usuário que significa o quanto o primeiro confia que os produtos recomendados pelo segundo são de fato desejáveis para ele ("eu acho que ele sabe do que eu gosto"). Possui 4 níveis: x,y,z,w

\paragraph{Amizade:} grau atribuído por um usuário para outro usuário que significa o quanto o primeiro conhece as preferências do segundo ("eu sei do que ele gosta"). Possui 4 níveis: x,y,z,w

\paragraph{Opinião:} grau atribuído por um usuário para um produto, que significa o quanto o usuário gosta do produto ("eu gosto desta coisa"). Tem 6 níveis, sendo o primeiro indicativo de "não conheço este produto".

\paragraph{Perfil:} conjunto de opiniões de um usuário.

\paragraph{Relevância:} opinião prevista pelo sistema para um produto ainda não avaliado pelo usuário.

\paragraph{Recomendação:} mensagem de um usuário a outro fazendo referência a um produto cadastrado no sistema. As recomendações podem ser avaliadas pelo usuário receptor em dois níveis, sendo um positivo e outro negativo.

\paragraph{Serendipidade:} mudança da opinião de um usuário sobre um produto de "não conheço este produto" para "eu gosto" após receber recomendação daquele produto.

\paragraph{Similaridade:} correlação entre dois produtos ou usuários.

\section{Descrição do Sistema}
\label{descricao}

Ao entrar no sistema, o usuário terá a opção de se cadastrar informando os seguintes dados: nome, login, e-mail, sexo, data de nascimento e foto.

Após o cadastro será  pedido que ele procure pessoas conhecidas na rede e os adicione como amigos da rede social indicando, em uma escala de 4 níveis, o quanto ela conhece a outra pessoa e o quanto ela confia nela para a recomendação de produtos em geral.

Cada pessoa adicionada como amiga receberá uma mensagem de solicitação de amizade. Caso a pessoa aceite o pedido de amizade, deverá avaliar da mesma forma o usuário solicitante.

Será pedido que o usuário busque no sistema produtos e os avalie. Nesse caso, para que o algoritmo de recomendação funcione da melhor maneira, é necessário que o usuário avalie 10 produtos que ele gosta e outros 10 que não gosta.

Cada produto será  apresentado através de uma foto, uma descrição e os comentários (reviews) que os outros usuários fizeram sobre ele. Será pedido que o usuário escolha em uma escala de 1 a 5 o quanto ele gosta daquele produto, caso ele o conheça. O usuário também poderá fazer comentários sobre o produto. Para encontrar os produtos para avaliá-los, ele buscará o produto pelo nome ou navegará nas listas de categorias.

Na tela principal (quando o usuário está logado), serão apresentados os seguintes itens:

\begin{itemize}
    \item Informações pessoais do usuário
    \item Lista resumida de amigos
    \item Listas de produtos recomendados pelo sistema
    \item Listas de produtos recomendados por outros usuários
    \item Área de busca básica de produtos
\end{itemize}

As seguintes listas de recomendações estarão disponíveis na tela principal:

\begin{itemize}
    \item Recomendações diretas
    \item Recomendações do sistema
    \item Produtos mais bem avaliados/novos
\end{itemize}

As recomendações do sistema podem ser feitas segundo os seguintes algoritmos:

\begin{itemize}
    \item Recomendação por similaridade entre perfis
    \item Recomendação por similaridade entre produtos
    \item Recomendação pelo parâmetro de "confiança"
\end{itemize}

A explicação dos algoritmos está na seção \ref{algoritmos}.

Para realizar uma recomendação, o usuário entra na descrição do produto e escolhe para quais amigos ele o recomendará.

Ao receber uma recomendação, o usuário deverá avaliar se gostou ou não da recomendação e avaliar o produto em uma escala de 1 a 5, sendo 1 não gostar e 5 gostar muito. Caso o usuário não conheça o produto, informar isso e não avaliá-lo. Há a possibilidade do usuário gostar da recomendação mesmo desconhecendo o produto.

Ao informar se conhece ou não o produto, a serendipidade poderá ser calculada. Sabendo se o usuário aprovou ou rejeitou a recomendação, o sistema atualiza o parâmetro de confiança entre o que recomenda e o que recebeu a recomendação. As recomendações diretas exibidas serão ordenadas de acordo com este parâmetro, portanto, se ele for muito baixo, a recomendação não terá muito destaque frente às outras. Ao avaliar o produto, o perfil do usuário é atualizado.

Caso o mesmo produto seja recomendado por mais de uma pessoa, ele aparece apenas uma vez na lista mantendo a ordem de relevância. O parâmetro de confiança do usuário em relação a cada pessoa que recomendou o produto será atualizado.

O usuário que recomendou o produto poderá ver se a recomendação foi aceita ou não e qual foi a avaliação dada pelo usuário recomendado.

O parâmetro de confiança entre os usuários não é exposto em nenhum momento no sistema.

\section{Algoritmos de recomendação}
\label{algoritmos}

Os três algoritmos básicos de recomendação são parecidos e seguem a seguinte estrutura.

Para cada item do sistema não avaliado pelo usuário a quem se deseja recomendar itens, é calculada a "pontuação" desse item para este usuário. Os itens são ordenados segundo esta pontuação e os "top" itens são mostrados ao usuário. Para cada algoritmo, o que muda é o parâmetro usado como pontuação.

\subsection{Recomendação por similaridade entre perfis}

Neste caso a pontuação de um item é a média aritmética ponderada entre os \textit{ratings} (avaliações) do item pelos usuários e a similaridade entre estes usuários e aquele usuário a quem se deseja fazer a recomendação do item.

A similaridade entre dois usuários é dada pela coeficiente de correlação de Pearson calculado sobre a avaliação feita por estes usuários aos mesmos itens. Usuários que avaliarem os mesmos itens da mesma maneira (notas parecidas) são considerados semelhantes.

\subsection{Recomendação por similaridade entre itens}

Neste caso a pontuação de um item é a média aritmética ponderada entre os \textit{ratings} dos itens já avaliados pelo usuário a quem se deseja recomendar e a similaridade entre esses itens e o item já avaliado.

A similaridade entre dois itens é dada pelo coeficiente de correlação de Pearson calculado sobre a avaliação feita pelos usuários aos dois itens. Se os usuários em geral avaliarem os dois itens de forma parecida (se quem deu nota alta/baixa para um item também deu nota alta/baixa para o outro), então os dois itens são considerados similares.

\subsection{Recomendação através do parâmetro de confiança}

Este é o algoritmo que precisa ser avaliado no experimento, pois sua idéia é relativamente nova na literatura.

Este algoritmo é idêntico ao algoritmo de similaridade entre perfis, mas substituindo a similaridade entre usuários pelo parâmetro de confiança entre o usuário a quem se deseja recomendar e os demais usuários.

A forma como o parâmetro de confiança varia já foi explicada na seção \ref{descricao} acima e se baseia na avaliação positiva ou negativa das recomendações diretas feitas ao usuário.

\section{Parâmetros medidos no sistema}

\subsection{Erro na avaliação prevista do produto}

Medir a diferença entre a avaliação do produto prevista pelo sistema.

\subsection{Taxa de rejeição de recomendações}

Este parâmetro será  medido sobre as recomendações feitas pelo sistema e também pelas recomendações feitas por outros usuários.

\subsection{Taxa de serendipidade}

Taxa de recomendações de produtos previamente desconhecidos ao usuário que ele gostou. Este parâmetro também será medido sobre as recomendações feitas pelo sistema e as feitas por outros usuários.

\subsection{Análise das medidas}

Nas medidas calculadas sobre as recomendações feitas por outros usuários, os graus de confiança e de amizade serão analisados posteriormente para verificar a influência sobre o parâmetro.

\end{document}
