%
%  untitled
%
%  Created by leobessa on 2009-06-21.
%  Copyright (c) 2009 __MyCompanyName__. All rights reserved.
%
\documentclass[]{article}

\usepackage{pst-gantt}
\usepackage{lscape}


% Use utf-8 encoding for foreign characters
\usepackage[utf8]{inputenc}

% Setup for fullpage use
\usepackage{fullpage}

% Uncomment some of the following if you use the features
%
% Running Headers and footers
%\usepackage{fancyhdr}

% Multipart figures
%\usepackage{subfigure}

% More symbols
%\usepackage{amsmath}
%\usepackage{amssymb}
%\usepackage{latexsym}

% Surround parts of graphics with box
\usepackage{boxedminipage}

% Package for including code in the document
\usepackage{listings}

% If you want to generate a toc for each chapter (use with book)
\usepackage{minitoc}

% This is now the recommended way for checking for PDFLaTeX:
\usepackage{ifpdf}

%\newif\ifpdf
%\ifx\pdfoutput\undefined
%\pdffalse % we are not running PDFLaTeX
%\else
%\pdfoutput=1 % we are running PDFLaTeX
%\pdftrue
%\fi

\ifpdf
\usepackage[pdftex]{graphicx}
\else
\usepackage{graphicx}
\fi
\title{Plano do Projeto de Formatura}
\author{Allan Douglas, Leonardo Bessa e Thiago Andrade  }

\date{23/06/2009}

\begin{document}

\ifpdf
\DeclareGraphicsExtensions{.pdf, .jpg, .tif}
\else
\DeclareGraphicsExtensions{.eps, .jpg}
\fi

\maketitle

%\begin{abstract}\end{abstract}

\section{Objetivo} % (fold)
\label{sec:objetivo}
% Apresentar de forma precisa e concisa o objetivo do projeto.

 O projeto consiste no desenvolvimento de um sistema de recomendação acoplado a redes sociais e compatível com padrões utilizados na web semântica.

% section objetivo (end)

\section{Justificativa} % (fold)
\label{sec:objetivos_e_justificativas}

Sistemas de Recomendação sugerem aos usuários itens que eles possam gostar baseado no comportamento prévio do usuário. Ao oferecer valor aos usuários, fazendo suposições pertinentes sobre o tipo de objetos em que estão interessados, é possível conquistar a confiança deles. A vantagem para os usuários é a facilidade de encontrar a informação sem ter a árdua tarefa de procurá-la.

As redes sociais online têm modificado a forma com que as empresas utilizam a comunicação para o comércio. Pessoas estão utilizando a Web para encontrar outras pessoas com interesses similares, fazer compras de forma mais eficiente, aprender sobre produtos e serviços e reclamar sobre produtos malfeitos e serviços pobres.

A Web está rapidamente se tornando a mídia mais importante para o marketing. A tendência é que as pessoas, cada vez mais bloqueiem os anúncios indesejados e queiram ter a capacidade de encontrar os produtos relevantes no momento adequado. É nesse contexto que surge a necessidade de uma plataforma que facilite a colaboração e que permita a criação e classificação de conteúdo pelos consumidores, de forma a permitir uma escolha mais inteligente dos melhores produtos e serviços e ao mesmo tempo criando uma mecanismo de feedback para as empresas interessadas.

Devido à grande variedade atual de produtos e serviços, as pessoas têm cada vez mais dificuldade nas suas escolhas e na argumentação sobre a possível decisão. Quanto mais produtos similares de fabricantes diferentes ou serviços realizados por diversos prestadores, mais as pessoas se vêem desnorteadas e sem saber se a decisão realizada foi a mais correta.

% section objetivos_e_justificativas (end)

\section{Gerentes de projeto}
\label{sec:gerentes_de_projeto}

Os integrantes do grupo do projeto de formatura são os gerentes do projeto e possuem a mesma autoridade.

% section gerentes_de_projeto (end)

\section{Partes interessadas}
\label{sec:partes_interessadas}

Uma das partes interessadas são as pessoas que têm vontade de obter informações sobre produtos a partir de seus amigos presentes na rede social, os quais elas podem demonstrar confiança para receber recomendações.
Também são interessadas no projeto as pessoas que queiram realizar recomendações de um produto e expressar a sua opinião sobre o mesmo.

% section partes_interessadas (end)

\section{Funcionalidades principais} % (fold)
\label{sec:funcionalidades_principais}

Abaixo estão sumarizadas as principais funcionalidades do sistema acompanhadas das descrições em formato de \textit{user stories}:
% referencia faltando!!!

\begin{itemize}

	\item Recomendação de produtos baseada em preferência do usuário
	\subitem Como usuário, eu quero receber uma lista de produtos que eu provavelemente goste para que eu não precise filtrar os itens que me interessam.

	\item Permitir que o usuário avalie um produto
  \subitem Como usuário, quero fazer a minha avaliação de produtos para que outros tenham conhecimento da minha opinião.

	\item Permitir que o usuário envie uma recomendação a outro usuário
  \subitem Como recomendador, quero enviar uma recomendação de produto para outra pessoa para que ela conheça a minha opinião sobre este item.

	\item Exposição das avaliações em formatos abertos
  \subitem Como agente web, quero obter as avaliações dos usuários em formato padronizado (e aberto) para que possa interpretar estas avaliações.

    \item Cadastro de novos produtos
    \subitem Como usuário, quero poder cadastrar um novo produto que ainda não está cadastrado.
    
    \item Importação de conteúdo em formato aberto
    \subitem Como usuário, quero poder informar uma fonte de conteúdo em formato aberto para que possa adicionar dados externos ao repositório.

    \item Explicação da recomendação
    \subitem Como usuário, quero obter a explicação sobre as recomendações recebidas para analisar criticamente a recomendação e me motivar a avaliar o produto.

    \item Visualização das avaliações recentes dos amigos
    \subitem Como usuário, quero estar ciente sobre as últimas avaliações dos meus amigos para que eu possa saber o que acontece no meu contexto social e conheça novos produtos.

    \item Feedback de recomendação
    \subitem Como recomendador, quero receber feedback sobre as recomendações que enviei para me incentivar a fazer melhores recomendações.

	
\end{itemize}

% section funcionalidades_principais(end)

\section{Requisitos não-funcionais} % (fold)

Os principais requisitos não-funcionais do sistema são:

\begin{itemize}

    \item Interface web compatível as últimas versões dos \textit{browsers} Internet Explorer, Mozilla Firefox e Safari.
    
    \item \textit{Backend} compatível com servidores Linux.

    \item Tempo médio de resposta menor que 2 segundos para 10 usuários simultâneos quando executado em um servidor com processador Intel Core 2 Duo T7250 ou superior e 2 GB de memória RAM. % esta é a configuração do meu notebook -- Allan

\end{itemize}
% section RNF (end)

\section{Estrutura Analítica do Projeto} % (fold)
\label{sec:estrutura_analitica_do_projeto}

\begin{enumerate}
		
	\item Pesquisa Bibliográfica
	\begin{enumerate}
		\item Sistema de Recomendação
		\begin{enumerate}
			\item Baseados em Conteúdo
			\item Filtragem Colaborativa
			\item Baseados em Confiança
		\end{enumerate}
		\item Web Semântica
		\begin{enumerate}
		  \item Microformatos
			\item Resource Description Framework (RDF)
			\item Web Ontology Language (OWL)
		\end{enumerate}
		\item Web Social
		\begin{enumerate}
		  \item Padrões de Projeto Sociais
		\end{enumerate}
	\end{enumerate}
	
	\item  Monografia
	\begin{enumerate}
		\item Introdução
		\begin{enumerate}
		  \item Objetivo
		  \item Motivação
		  \item Organização
		\end{enumerate}
		\item Aspectos Conceituais
		\begin{enumerate}
		  \item Web Social
		  \item Web Semântica
		  \item Sistemas de Recomendação
		\end{enumerate}
		\item Especificação do Projeto de Formatura
		\begin{enumerate}
		  \item Características Funcionais
		  \item Detalhamento do Software
		  \item Especificação de Requisitos
		\end{enumerate}
		\item Metodologia
		\begin{enumerate}
		  \item Concepção
		  \item Implementação
		  \item Testes
		\end{enumerate}
		\item Projeto e Implementação
		\item Testes e Avaliação
		\item Considerações Finais
		\begin{enumerate}
		  \item Resultados Atingidos
		  \item Resultados Não Atingidos (com justificativa)
		  \item Contribuições
		  \item Perspectivas de Continuidade
		  \item Comentários individuais
		\end{enumerate}
	\end{enumerate}
		
	\item Desenvolvimento
	\begin{enumerate}
		\item Sistema de recomendação
		\begin{enumerate}
			\item Captura e Parsing de documentos semânticos
			\item Algoritmo de recomendação
			\item Interface de consulta de recomendações
		\end{enumerate}
		\item Repositório de documentos semânticos
		\begin{enumerate}
		  \item Inserção/Atualização de Produtos e Avaliações
		  \item Exposição de Produtos e Avalições
		\end{enumerate}
		\item Front-End
		\begin{enumerate}
			\item Aplicação para rede social
			\item Camada de Apresentação
			\item Integração com a base de documentos
			\item Integração com o Sistema de recomendação
		\end{enumerate}
	\end{enumerate}
	
\end{enumerate}
% section estrutura_analítica_do_projeto (end)

\section{Cronograma} % (fold)
\label{sec:cronograma}

\begin{itemize}
  \item 13 de Julho de 2009
  \begin{itemize}
    \item Entrega das seções sobre RDF e Microformatos
    \item Entrega Repositório de documentos Semânticos
  \end{itemize}
  
  \item 27 de Julho de 2009
  \begin{itemize}
    \item Entrega da seção sobre OWL
    \item Funcionalidade de Captura e Parsing de documentos semânticos
  \end{itemize}
  
  \item 10 de Agosto de 2009
  \begin{itemize}
    \item Apresentação do Calendário do Projeto de Formatura
    \item Entrega do Capítulo de Aspectos Conceituais
  \end{itemize}
  
  \item 24 de Agosto de 2009
  \begin{itemize}
    \item Finalização do Algoritmo de recomendação
    \item Entrega do Capítulo de Especificação do Projeto de Formatura
  \end{itemize}
  
  \item 7 de Setembro de 2009
  \begin{itemize}
    \item Entrega Interface de consulta de recomendações
    \item Apresentação da Evolução do Projeto e entrega de documento.
  \end{itemize}
  
  \item 21 de Setembro de 2009
  \begin{itemize}
    \item Entrega do Capítulo de Metodologia
    \item Implantação do Front-End
  \end{itemize}
  
  \item 5 de Outubro de 2009
  \begin{itemize}
    \item Entrega do Capítulo de Projeto e Implementação
    \item Formulação de Testes e de Avaliação
  \end{itemize}
  
  \item 19 de Outubro de 2009
  \begin{itemize}
    \item Avaliação dos Resultados parciais
  \end{itemize}
  
  \item 2 de Novembro de 2009
  \begin{itemize}
    \item Entrega do Capítulo de Testes e Avaliação
    \item Apresentação da Evolução do Projeto e entrega de documento.
    \item Entrega do Pôster e Press Release, Página Internet
  \end{itemize}
  
  \item 16 de Novembro de 2009
  \begin{itemize}
    \item Capítulo de Considerações Finais
    \item Seção sobre Organização na Introdução
  \end{itemize}
  
  \item 30 de Novembro de 2009
  \begin{itemize}
    \item Entrega da Monografia
  \end{itemize}
  
  \item 8 de Dezembro de 2009
  \begin{itemize}
    \item Apresentação e Demonstração Prática Perante Banca
  \end{itemize}
\end{itemize}

\subsection{Gráfico de Gantt}

A seguir está um gráfico de Gantt com as principais atividades. O Mês 1 é Julho e o Mês 6 é Dezembro.

\definecolor{LightCyan}   {rgb}{0.88,1.,1.}
\definecolor{Orange}      {rgb}{1.,0.65,0.}
\definecolor{PaleGreen}   {rgb}{0.6,0.98,0.6}
\definecolor{Pink}        {rgb}{1.,0.75,0.8}
\psset{gradmidpoint=0,fillstyle=gradient,gradbegin=LightCyan,gradend=white}
\newpsstyle{TaskStyleA}{gradbegin=cyan,gradend=blue}
\newpsstyle{TaskStyleB}{gradbegin=red,gradend=Pink}
\newpsstyle{TaskStyleC}{gradbegin=yellow,gradend=Orange}
\newpsstyle{TaskStyleD}{gradbegin=green,gradend=PaleGreen}
\begin{center}
\begin{PstGanttChart}[yunit=2.5,xunit=3.5,ChartUnitIntervalName=Mês,
                      ChartUnitBasicIntervalName=Week,TaskUnitType=Week,
                      TaskUnitIntervalValue=4,
                      TaskOutsideLabelMaxSize=14,ChartShowIntervals]{18}{24}
  \psset{gradangle=90,TaskStyle=TaskStyleA}

  \PstGanttTask[TaskOutsideLabel={Seções sobre RDF e Microformatos}]{1}{2}
  \PstGanttTask[TaskOutsideLabel={Repositório de documentos Semânticos}]{1}{2}
  \PstGanttTask[TaskOutsideLabel={Seção sobre OWL}]{2}{2}
  \PstGanttTask[TaskOutsideLabel={Captura e Parsing de documentos semânticos}]{2}{2}
  \PstGanttTask[TaskOutsideLabel={Capítulo de Aspectos Conceituais}]{3}{3}
  \PstGanttTask[TaskOutsideLabel={Algoritmo de recomendação}]{1}{6}
  \PstGanttTask[TaskOutsideLabel={Capítulo Especificação}]{3}{4}
  \PstGanttTask[TaskOutsideLabel={Interface de consulta de recomendações}]{5}{4}
  \PstGanttTask[TaskOutsideLabel={Documento de Evolução do Projeto}]{8}{1}
  \PstGanttTask[TaskOutsideLabel={Capítulo de Metodologia}]{9}{2}
  \PstGanttTask[TaskOutsideLabel={Implantação do Front-End}]{5}{6}
  \PstGanttTask[TaskOutsideLabel={Capítulo de Projeto e Implementação}]{9}{4}
  \PstGanttTask[TaskOutsideLabel={Formulação de Testes e Avaliação}]{9}{4}
  \PstGanttTask[TaskOutsideLabel={Avaliação dos Resultados parciais}]{13}{2}
  \PstGanttTask[TaskOutsideLabel={Capítulo de Testes e Avaliação}]{15}{2}
  \PstGanttTask[TaskOutsideLabel={Capítulo de Considerações finais}]{16}{2}
  \PstGanttTask[TaskOutsideLabel={Finalização da Monografia}]{17}{3}
  \PstGanttTask[TaskOutsideLabel={Apresentação e Demostração Prática}]{18}{4}

  
\end{PstGanttChart}
\end{center}

% section cronograma (end)


\section{Custos} % (fold)
\label{sec:custos}

\begin{description}
  \item[Desenvolvimento]
  
  Cada integrante trabalhará quatro horas por semana durante três meses, totalizando 144 horas de desenvolvimento. Considerando o valor da hora paga a um desenvolvedor WEB de R\$ 15,00, o custo de desenvolvimento totaliza em \textbf{R\$ 2.160,00}.
  
  \item[Monografia] 
  
  Cada integrante gastará quatro horas por semana durante cinco meses escrevendo a monografia. Dessa forma, podemos contabilizar o gasto utilizando como base o piso salarial de um engenheiro de computação (R\$ 3.720,00), ou seja R\$ 23,25/hora, totalizando um custo de \textbf{R\$ 5.580,00} para a produção da monografia.
  
  \item[Total] \textbf{R\$ 7.740} = \textbf{R\$ 2.160,00} + \textbf{R\$ 5.580,00}.
  
  
\end{description}

% section custos (end)


\section{Estrutura Analítica de Riscos} % (fold)
\label{sec:estrutura_analitica_de_riscos}

\begin{itemize}
  
  \item Tecnologia
  \begin{itemize}
    \item Requisitos
    %\begin{description}
      %\item[Incerteza do Escopo]
      %\item[Complexidade] 
    %\end{description}
    \item Aplicação
    \begin{description}
      \item[Conjunto de Habilidades Pessoais \& Experiência] Caso os integrantes tenham dificuldades para utilizar as tecnologias propostas no projeto.
      \item[Recursos Físicos] Caso algum dos computadores pessoais ou servidores parem funcionar corretamente, as entregas podem sofrem atrasos significativos.
    \end{description}
    \item Desempenho
    \begin{description}
      \item[Tempo de Resposta] Caso o tempo de resposta do sistema seja muito maior do que o aceitável pelo usuário, o sistema corre um grande risco de falhar.
      \item[Número de Usuários] Caso o sistema não consiga trabalhar adequadamente com um número mínimo de usuários.
    \end{description}
  \end{itemize}
  
  \item Gerencial
  \begin{itemize}
    \item Corporativo
    \begin{description}
      \item[Greve] A greve que ocorre em alguns departamentos da USP, tem mínimas chances de afetar o andamento das atividades da Escola Politécnica.
      \item[Financeira] O projeto tem baixa probabilidade de necessitar de recursos financeiros externos.
    \end{description}
    \item Cliente
    \begin{description}
      \item[Cultura] É possível que os usuários não se adaptem a cultura de fazer recomendações e buscar por produtos recomendados. Caso isso aconteça, o projeto corre sério risco de fracasso. 
    \end{description}
  \end{itemize}
  
  \item Externos
  \begin{itemize}
    
    \item Cultural
    \begin{description}
      \item[Informações inverídicas] Existe a possibilidade de usuários utilizarem o sistema para disseminar calúnias e difamações. Neste caso, os criadores do sistema podem ser responsabilizados legalmente pela exposição dessas informações.
      \item[Usuários mal intencionados] Usuários podem agir de má fé, enviando resenhas com palavras de baixo calão ou até mesmo conteúdo obsceno. Este tipo de atitude pode comprometer a imagem do sistema e afastar usuários.
    \end{description}
    
    \item Política
    \begin{description}
      \item[Aprovação do Aplicativo na Rede Social] Pode ser que a aplicação desenvolvida demore para ser aprovada pelas redes sociais externas (Orkut e MySpace), ou seja rejeitada para entrar no diretório oficial de aplicativos delas.
    \end{description}
  \end{itemize}
  
\end{itemize}

% section estrutura_analítica_de_riscos (end)

\section{Análise SWOT} % (fold)
\label{sec:analise_swot}
\begin{center}
\begin{tabular}{|l|l|}
\hline
\textbf{Forças} & \textbf{Fraquezas}\\
\hline
Conhecimento Técnico & Organização dos Horários\\
Orientação & Criação de Interfaces de Usuário\\
\hline
\textbf{Oportunidades} & \textbf{Ameaças}\\
\hline
Adoção da Web Semântica & Não aceitação\\
Redes Sociais &  \\
\hline

\end{tabular}
\end{center}

\begin{description}
  \item Forças
  \begin{description}
    \item[Conhecimento Técnico] Toda a equipe está familiarizada com a área técnica abordada na elaboração dos trabalho, sendo que muitas das técnicas de projeto e implantação são ensinadas no curso de engenharia de computação. Além disso, todos já realizaram estágios na área de desenvolvimento de software durante a sua carreira acadêmica.
    
    \item[Orientação] 
    
     O projeto é orientado pelo Prof. Dr. Jaime Simão Sichman que obteve formação de Livre-Docência em Inteligência Artificial. Além disso, o projeto conta com a co-orientação da Prof. Dra. Lucia Filgueiras que atua na área de Análise e Projeto de Interfaces Homem-Computador.
     
  \end{description}
  
  \item Fraquezas
  \begin{description}
    \item[Organização dos Horários] A grade horária do curso de Engenharia de Computação, juntamente com os horários do estágios dificultam a organização do tempo para realização de tarefas em grupo. Isso afeta o tempo das reuniões presenciais, bem como algumas práticas da modelagem ágil, como programação pareada.
    \item[Criação de Interfaces de Usuário] A equipe apresenta pouca experiência no desenvolvimento de interfaces de usuário ricas para Web.
  \end{description}
  
  \item Oportunidades
  \begin{description}
    \item[Adoção da Web Semântica] O sistema de recomendação obtem informações a partir de documentos semanticamente estruturados que descrevem as avaliações dos produtos. Se as empresas de e-commerce brasileiras adotarem os padrões da Web Semâtinca, os produtos vendidos por estas poderão fazer parte da base de dados do sistema desenvolvido, ampliando o seu alcançe.
    \item[Redes Sociais]
         No Brasil, apesar da maioria das empresas não investir em tecnologias de redes sociais, há vários usuários cadastrados em sites de relacionamento e que os acessam diariamente, contribuindo para a elaboração de conteúdo e comunicação via Internet. Essa utilização vem mostrando um crescimento acentuado nos últimos anos, sendo uma grande oportunidade, pois grandes empresas se deram conta disso no restante do mundo e não demorará muito para que as empresas brasileiras se dêem conta disso também, sendo que algumas delas já investem nesse tipo negócio.
  \end{description}
  
  \item Ameaças
  \begin{description}
    \item[Não Aceitação] A aceitação do sistema pode vir de apenas alguns usuários. Caso isso aconteça, o sistema pode continuar sendo utilizado como um canal de divulgação de produtos. No entanto a realização de testes para avaliar a efetividade do sistema fica comprometida.
  \end{description}
\end{description}

% section análise_swot (end)

%\bibliographystyle{plain}
%\bibliography{}
\end{document}
